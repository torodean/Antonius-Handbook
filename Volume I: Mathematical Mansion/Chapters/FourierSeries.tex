\chapter{Fourier Series} \index{Fourier Series}
\thispagestyle{fancy}
The computation of the (usual) Fourier series is based on the integral identities 
\begin{multicols}{2}\noindent
\begin{align}
&\int_{-\pi}^{\pi}\sin(mx)\sin(nx)dx=\pi\delta_{mn} \\
&\int_{-\pi}^{\pi}\cos(mx)\cos(nx)dx=\pi\delta_{mn} \\
&\int_{-\pi}^{\pi}\sin(mx)\cos(nx)dx=0 
\end{align}
\begin{align}
&\int_{-\pi}^{\pi}\sin(mx)dx=0 \\
&\int_{-\pi}^{\pi}\cos(mx)dx=0 \\
&\delta_{mn} = \frac{1}{2\pi i}\oint_\gamma z^{m-n-1}dz
\end{align}
\end{multicols}
Using the method for a generalized Fourier series, the usual Fourier series involving sines and cosines is obtained by taking $f_1(x)=cosx$ and $f_2(x)=sinx$. Since these functions form a complete orthogonal system over $[-\pi,\pi]$, the Fourier series of a function $f(x)$ is given by (with $n\in \mathbb{N}$)
\begin{align}
f(x)&=\frac{1}{2}a_0+\sum_{n=1}^{\infty}a_n\cos(nx)+\sum_{n=1}^{\infty}b_n\sin(nx) \\
a_0&= \frac{1}{\pi} \int_{-\pi}^{\pi}f(x)dx \\
a_n&= \frac{1}{\pi} \int_{-\pi}^{\pi}f(x)\cos(nx)dx \\
b_n&= \frac{1}{\pi} \int_{-\pi}^{\pi}f(x)\sin(nx)dx
\end{align}
The notion of a Fourier series can also be extended to complex coefficients. 
\begin{align}
	f(x) &= \sum_{n=-\infty}^{\infty} A_n e^{inx} \hspace{0.5cm}\textrm{with}\hspace{0.5cm}
	A_n = \frac{1}{2\pi} \int_{-\pi}^{\pi}f(x)e^{-inx}dx
\end{align}
For a function f(x) periodic on an interval [-L,L] instead of [-pi,pi], a simple change of variables can be used to transform the interval of integration from [-pi,pi] to [-L,L]. Let 
\begin{align}
x &\equiv \frac{\pi x'}{L}  \Longleftrightarrow x'\equiv\frac{Lx}{\pi} \implies
dx = \frac{\pi dx'}{L}
\end{align}
