\section{Trigonometric Identities}\index{Trigonometric Identities}
\begin{multicols}{2}
Pythagorean identities:
\begin{align}
1 &= \sin^2(\theta)+\cos^2(\theta)\\
1 &= \sec^2(\theta)-\tan^2(\theta) \\
1 &= \csc^2(\theta)-\cot^2(\theta) \\
1 &= \cosh^2(\theta)-\sinh^2(\theta) \\
1 &= \textrm{sech}^2(\theta)+\tanh^2(\theta)
\end{align}
Sum-Difference Formulas:
\begin{align}
\sin(\theta \pm \phi) &= \sin(\theta)\cos(\phi)\pm \cos(\theta)sin(\phi) \\
\cos(\theta \pm \phi) &= \cos(\theta)\cos(\phi)\mp \sin(\theta)sin(\phi) \\
\tan(\theta \pm \phi) &= \frac{\tan(\theta)\pm \tan (\phi)}{1 \mp \tan(\theta)\tan(\phi)}
\end{align}
Double Angle formulas:
\begin{align}
\sin(2\theta) &= 2\sin(\theta)\cos(\theta) \\
\cos(2\theta) &= \cos^2(\theta)-\sin^2(\theta)\\
&= 2\cos^2(\theta)-1 \\
&= 1 - 2\sin^2(\theta) \\
\tan(2\theta) &= \frac{2 \tan(\theta)}{1-\tan^2(\theta)}
\end{align}
Power-Reducing/Half Angle Formulas:
\begin{align}
\sin^2(\theta) &= \frac{1-\cos(2\theta)}{2}\\
\cos^2(\theta) &= \frac{1+\cos(2\theta)}{2}\\
\tan^2(\theta) &= \frac{1-\cos(2\theta)}{1+\cos(2\theta)}
\end{align}
Phase changes follow:
\begin{align}
\sin(-\theta) &=-\sin(\theta) \\
\cos(-\theta) &= \cos(\theta) \\
\sin(\theta\pm \pi/2) &= \pm \cos(\theta) \\
\sin(\theta\pm \pi) &= - \sin(\theta) \\
\cos(\theta\pm \pi/2) &= \mp \sin(\theta) \\
\cos(\theta\pm \pi) &= - \cos(\theta) 
\end{align}
Half-angle formulas
\begin{align}
\sin\bigg(\frac{\theta}{2}\bigg)=(-1)^{\theta/(2\pi)}\sqrt{\frac{1-\cos(\theta)}{2}} \\
\cos\bigg(\frac{\theta}{2}\bigg)=(-1)^{(\theta+\pi)/(2\pi)}\sqrt{\frac{1+\cos(\theta)}{2}}
\end{align}
The \textbf{Weierstrass substitution}\index{Weierstrass substitution} makes use of the half-angle formulas 
\begin{align}
\cos(\theta)=\frac{1-\tan^2(\theta/2)}{1+\tan^2(\theta/2)} \\
\sin(\theta)=\frac{2\tan(\theta/2)}{1+\tan^2(\theta/2)}
\end{align}
Other relations and identities:
\begin{align}
	\cos(x)\cos(y) &= \frac{1}{2}[\cos(x-y)+\cos(x+y)] \\
	\sin(x)\sin(y) &= \frac{1}{2}[\cos(x-y)-\cos(x+y)]
\end{align}
\end{multicols}
The half angle identity for tangent.
\begin{align}
\tan\bigg(\frac{\theta}{2}\bigg)=(-1)^{x/\pi}\sqrt{\frac{1-\cos(\theta)}{1+\cos(\theta)}} = \frac{\sin(\theta)}{1+\cos(\theta)}=\frac{1-\cos(\theta)}{\sin(\theta)}=\frac{\tan(\theta)\sin(\theta)}{\tan(\theta)+\sin(\theta)}
\end{align}

Multiple-angle formulas are given by 
\begin{align}
\sin(nx)&= \sum_{k=0}^{n}{{n}\choose{k}}\cos^k(x)\sin^{n-k}(x)\sin\big((n-k)\pi/2 \big) \\
\cos(nx)&= \sum_{k=0}^{n}{{n}\choose{k}}\cos^k(x)\sin^{n-k}(x)\cos\big((n-k)\pi/2 \big)
\end{align}

Other identities
\begin{align}
\cos(\theta)cos(\phi) &=\frac{1}{2}[\cos(\theta+\phi)+\cos(\theta-\phi)] \\
\sin(\theta)sin(\phi) &=\frac{1}{2}[\cos(\theta-\phi)-\cos(\theta+\phi)] \\
\sin(\theta)cos(\phi) &=\frac{1}{2}[\sin(\theta+\phi)+\sin(\theta-\phi)] \\
\cos(\theta)+\cos(\phi)&= 2\cos\bigg( \frac{\theta+\phi}{2}\bigg)\cos\bigg( \frac{\theta-\phi}{2}\bigg) \\
\cos(\theta)-\cos(\phi)&= 2\sin\bigg( \frac{\theta+\phi}{2}\bigg)\sin\bigg( \frac{\theta-\phi}{2}\bigg)
\end{align}

