\section{Arbitrary Orthogonal Curvilinear Coordinates}\index{Orthogonal Coordinates}
A coordinate system composed of intersecting surfaces. If the intersections are all at right angles, then the curvilinear coordinates are said to form an orthogonal coordinate system. The scale factors are $h_i$,
\begin{align}
	\vec{a}_i &\equiv\frac{\partial \vec{r}}{\partial e_i} =  \frac{\partial x}{\partial e_i} \hat{x} + \frac{\partial y}{\partial e_i}\hat{y} + \frac{\partial x}{\partial e_i} \hat{z} = h_i \hat{e}_i = |\vec{a}_i| \hat{e}_i \\
	h_i &\equiv \left|\frac{\partial \vec{r}}{\partial e_i}\right|=|\vec{a}_i| = \sqrt{\frac{\partial x}{\partial e_i}+\frac{\partial y}{\partial e_i}+\frac{\partial z}{\partial e_i} } \\ \hat{e}_i &= \frac{1}{h_1}\frac{\partial \vec{r}}{\partial e_i}=\frac{\vec{a}_i}{|\vec{a}_i|}
\end{align}
The line element $d\vec{s}$ is determined by
\begin{align}
	d\vec{s}\equiv 
	d\vec{x}+d\vec{y}+d\vec{z} \equiv
	\vec{a}_1de_1+\vec{a}_2de_2+\vec{a}_3de_3
\end{align}
From this, $ds^2$ is given by
\begin{align}
	ds^2= d\vec{s}\cdot d\vec{s} = dx^2+dy^2+dz^2= h_1^2de_1^2+h_2^2de_2^2+h_3^2de_3^2
\end{align}
The differential vector and volume elements are therefore
\begin{align}
	d\vec{r} &= h_1 du_1 \hat{u}_1+h_2 du_2 \hat{u}_2+h_3 du_3 \hat{u}_3 \\
	dV &=h_1h_2h_3 du_1du_2du_3 = \left|\frac{\partial(x,y,z)}{\partial(u_1,u_2,u_3)}\right|du_1du_2du_3
\end{align}
The gradient in arbitrary curvilinear coordinates such that the gradient theorem is preserved:
\begin{align}
	\nabla f = \frac{1}{h_1}\frac{\partial f}{\partial x_1}\hat{x}_1+\frac{1}{h_2}\frac{\partial f}{\partial x_2}\hat{x}_2+\frac{1}{h_3}\frac{\partial f}{\partial x_3}\hat{x}_3
\end{align}
The divergence in arbitrary curvilinear coordinates such that the divergence theorem is preserved:
\begin{align}
	\nabla \cdot \vec{v} = \frac{1}{h_1h_2h_3}\left[\frac{\partial v_1}{\partial x_1}h_2h_3+\frac{\partial v_2}{\partial x_2}h_1h_3+\frac{\partial v_3}{\partial x_3}h_1h_2\right]
\end{align}
The Laplacian\index{Laplacian} for a scalar function $\phi$ (where the $h_i$ are the scale factors of the coordinate system - Weinberg 1972, p. 109; Arfken 1985, p. 92 \cite{bib:Wolfram}) is a scalar differential operator defined by
\begin{align}
	\nabla^2 \phi = \frac{1}{h_1h_2h_3} \bigg[	\frac{\partial}{\partial u_1} \bigg( \frac{h_2h_3}{h_1} \frac{\partial}{\partial u_1} \bigg) + \frac{\partial}{\partial u_2} \bigg( \frac{h_1h_3}{h_2} \frac{\partial}{\partial u_2} \bigg) + \frac{\partial}{\partial u_3} \bigg( \frac{h_1h_2}{h_3} \frac{\partial}{\partial u_3} \bigg) \bigg] \phi
\end{align}
The form of the Laplacian in several common coordinate systems (cartesian, cylindrical, parabolic, parabolic cylindrical, spherical and oblate spheroidal respectively) are
\begin{align}
	\nabla^2 f &= \frac{\partial^2 f}{\partial x^2}+ \frac{\partial^2 f}{\partial y^2}+ \frac{\partial^2 f}{\partial z^2} \\
	\nabla^2 f &=\frac{1}{r}\frac{\partial}{\partial r}\bigg(r \frac{\partial f}{\partial r}\bigg)+\frac{1}{r^2}\frac{\partial^2 f}{\partial \theta^2}+\frac{\partial^2 f}{\partial z^2} \\
	\nabla^2 f &= \frac{1}{uv(u^2+v^2)}\bigg[\frac{\partial}{\partial u}\bigg( uv\frac{\partial f}{\partial u} \bigg) + \frac{\partial}{\partial v}\bigg(u v \frac{\partial f}{\partial v}\bigg)\bigg]+\frac{1}{v^2u^2}\frac{\partial^2 f}{\partial \theta^2} \\
	\nabla^2 f &= \frac{1}{u^2+v^2}\bigg(\frac{\partial^2 f}{\partial u^2}+\frac{\partial^2 f}{\partial v^2} \bigg)+\frac{\partial^2 f}{\partial z^2} \\
	\nabla^2 f &= \frac{1}{r^2} \frac{\partial}{\partial r}\bigg(r^2 \frac{\partial f}{\partial r} \bigg)+\frac{1}{r^2 \sin^2 \phi}\frac{\partial^2 f}{\partial \theta^2} +\frac{1}{r^2\sin\phi}\frac{\partial}{\partial\phi}\bigg(\sin\phi \frac{\partial f}{\partial \phi} \bigg) \\
	\nabla^2 f&= \frac{1}{a^2(\zeta^2+\xi^2)}\left[\frac{\partial}{\partial \zeta}\left((1+\zeta^2)\frac{\partial f}{\partial \zeta}\right) +\frac{\partial}{\partial \xi}\left((1-\xi^2)\frac{\partial f}{\partial \xi}\right) \right] +\frac{1}{a^2(1+\zeta^2)(1-\xi^2)} \frac{\partial^2 f}{\partial \phi^2}.
\end{align}
The \textbf{curl}\index{Curl} can be similarly defined in arbitrary orthogonal curvilinear coordinates as
\begin{align}
	\nabla \times \vec{F} &\equiv \frac{1}{h_1h_2h_3}
	\begin{vmatrix}
		h_1\hat{e}_1 & h_2\hat{e}_2 & h_3\hat{e}_3  \\ 
		\frac{\partial}{\partial e_1} & \frac{\partial}{\partial e_2} & \frac{\partial}{\partial e_3}  \\ 
		h_1F_1 & h_2F_2 & h_3F_3  
	\end{vmatrix} \\
	&= \frac{1}{h_2h_3}\left[\frac{\partial}{\partial u_2}(h_3F_3)-\frac{\partial}{\partial u_3}(h_2F_2)\right]\hat{u}_1+\frac{1}{h_1h_3}\left[\frac{\partial}{\partial u_3}(h_1F_1)-\frac{\partial}{\partial u_1}(h_3F_3)\right]\hat{u}_2 \nonumber\\
	&\hspace{6.72cm}+\frac{1}{h_1h_2}\left[\frac{\partial}{\partial u_1}(h_2F_2)-\frac{\partial}{\partial u_2}(h_1F_1)\right]\hat{u}_3.
\end{align} 

The \textbf{Jacobian}\index{Jacobian} is defined as the determinant of a matrix of partial derivatives \cite{bib:StellarAstrophysics},
\begin{align}
	\frac{\partial(a,b)}{\partial(c,d)} \equiv \begin{vmatrix}
		\left(\frac{\partial a}{\partial c}\right)_d & \left(\frac{\partial a}{\partial d}\right)_c \\
		\left(\frac{\partial b}{\partial c}\right)_d & \left(\frac{\partial b}{\partial d}\right)_c
	\end{vmatrix} =\left(\frac{\partial a}{\partial c}\right)_d\left(\frac{\partial b}{\partial d}\right)_c - \left(\frac{\partial a}{\partial d}\right)_c\left(\frac{\partial b}{\partial c}\right)_d.
\end{align}
By the above definition, we can show the relations,
\begin{align}
	\frac{\partial(b,a)}{\partial(c,d)}=-\frac{\partial(a,b)}{\partial(c,d)} \andspace{1cm} \frac{\partial(a,b)}{\partial(c,d)}=-\frac{\partial(a,b)}{\partial(d,c)}.
\end{align}
It then follows directly that
\begin{align}
	\frac{\partial(a,s)}{\partial(c,s)} = \left(\frac{\partial a}{\partial c}\right)_s \andspace{1cm} \frac{\partial(a,b)}{\partial(a,b)} = 1 \andspace{1cm} \frac{\partial(a,b)}{\partial(c,d)}\frac{\partial(c,d)}{\partial(s,t)} = \frac{\partial(a,b)}{\partial(s,t)}.
\end{align}