\chapter{Integrals}
\thispagestyle{fancy}
Basic indefinite integrals. Add $c$ to each result for completion ($c=$constant).
\begin{multicols}{2}
\noindent
\begin{align}
	&\int \frac{dx}{a^2+x^2} = \frac{1}{a}\arctan\bigg(\frac{x}{a}\bigg) \\
	&\int \frac{dx}{\sqrt{a^2-x^2}}= \textrm{arcsin}\bigg(\frac{x}{|a|} \bigg) \\&\hspace{2.1cm}=\arctan\bigg(\frac{x}{\sqrt{a^2-x^2}} \bigg)\\
	&\int \frac{dx}{x+x^2} =\ln\bigg(\frac{x}{1+x} \bigg) \\
	&\int \frac{dx}{\sqrt{x^2-1}} = \textrm{arccosh}(x) \\
	&\int \frac{dx}{x\sqrt{x^2-1}}= \arccos\bigg(\frac{1}{x}\bigg) \\
	&\int \frac{dx}{(a^2+x^2)^{3/2}} = \frac{x}{a^2\sqrt{a^2+x^2}} \\
	&\int \frac{xdx}{(a^2+x^2)^{3/2}} = -\frac{1}{\sqrt{a^2+x^2}} \\
	&\int \frac{dx}{\sqrt{a^2+x^2}} = \textrm{arcsinh}\bigg(\frac{x}{a}\bigg) \\ &\hspace{2.1cm}= \ln\big|x+\sqrt{a^2+x^2}\big|
	\end{align}
	\begin{align}
	&\int \frac{xdx}{1+x^2} =\frac{1}{2}\ln(1+x^2) \\
	&\int \frac{dx}{1-x^2} = \textrm{arctanh}(x) \\
	&\int \frac{xdx}{\sqrt{1+x^2}}= \sqrt{1+x^2} \\
	&\int \frac{\sqrt{x}dx}{\sqrt{1-x}}= \arcsin(\sqrt{x})-\sqrt{x(1-x)} \\
	&\int \frac{x^2}{a^2+x^2} dx=\frac{a^2}{2}\sin^{-1}\left(\frac{x}{a}\right)-\frac{x}{2}\sqrt{a^2-x^2}\\
	&\int \frac{dx}{(a^2+x^2)^2} = \frac{x}{2a^2(x^2+a^2)}+\frac{1}{2a^3}\arctan\left(\frac{x}{a}\right)  \\
	&\int \frac{x^2dx}{(a^2+x^2)^2} = \frac{-x}{2(x^2+a^2)}+\frac{1}{2a}\arctan\left(\frac{x}{a}\right)  \\
	&\int \ln(x)=x\ln(x)-x
\end{align}
\end{multicols}
By use of the product rule $\frac{d}{dx}[f(x)g(x)]=f'(x)g(x)+f(x)g'(x)$, we can integrate both sides to arive at a useful formula for solving various integrals known as \textbf{integration by parts}\index{Integration by parts}
\begin{align}
\int f(x)g'(x)=f(x)g(x)-\int g(x)f'(x) \hspace{1cm}\textrm{ or }\hspace{1cm} \int udv=uv-\int vdu
\end{align} 
Generalizing this result gives us integration by parts as
\begin{align}
\int f^{(n)}(x)g(x) dx = g(x)f^{(n-1)}(x)-g'(x)f^{(n-2)(x)} + g''(x)f^{(n-3)}(x) - \cdots +(-1)^n \int f(x)g^{(n)}(x)dx
\end{align}
Some numerical results (with $\zeta$ being the usual zeta function) include
\begin{multicols}{2}
	\noindent
	\begin{align}
		&\int_{-\infty}^{\infty} \frac{x^2e^x}{(e^x+1)^2}dx=\frac{\pi^2}{3} \\
		&\int_{0}^{\infty} \frac{x}{e^{ax}+1}dx=\frac{\pi^2}{12a^2}
	\end{align}
	\begin{align}
	&\int_{0}^{\infty} \frac{\sqrt{x}dx}{e^x-1}=\frac{\sqrt{\pi}}{2}\zeta\left(\frac{3}{2}\right)\approx1.306\sqrt{\pi} 
	\end{align}
\end{multicols}

\invisiblesection{Brief Table of Integrals}
\includepdf[pages=1,pagecommand=\thispagestyle{fancy}, scale=0.8, pagecommand={\footnotetext{This "Brief Table of Integrals" is taken directly from Dennis G. Zill - A First Course in Differential Equations, 10th Ed. When time permits it will be re-created in an original format.}}]{Resources/IntegralTable}
\newpage

Exponential integrals
\begin{align}
	&\int_{-\infty}^{\infty} \frac{e^{-iax}}{(1+x^2)}dx = \pi e^{-|a|}
\end{align}
\begin{fancybox}[Trigonometric Integrals\index{Trigonometric integrals}]{1}
\begin{align}
	&\int \tan(x)dx = -\ln(\cos(x)) +c\\
	&\int \tanh(x)dx = \ln(\cosh(x))+c \\
	&\int \sin^2(x)dx = \frac{1}{2}\big(x-\sin(x)\cos(x)\big)+c = \frac{1}{4}\big(2x-\sin(2x)\big)+c\\
	&\int \cos^2(x)dx = \frac{1}{2}\big(x+\sin(x)\cos(x)\big)+c = \frac{1}{4}\big(2x+\sin(2x)\big)+c \\
	&\int \sin^2(x)\cos(x)dx = \frac{1}{3}\sin^3(x)+c \\
	&\int \cos^2(x)\sin(x)dx = -\frac{1}{3}\cos^3(x)+c \\
	&\int \sin^3(x)dx = -\frac{1}{3}\cos(x)\big(\sin^2(x)+2\big)+c \\
	&\int x\sin^2(x)dx = \frac{1}{4}\big(x^2-x\sin(2x)-\frac{1}{2}\cos(2x)\big)+c\\
	&\int x^2\sin^2(x)dx =\frac{x^3}{6}-\bigg(\frac{x^2}{4}-\frac{1}{8}\bigg)\sin(2x)-\frac{x}{4}\cos(2x)+c \\
	&\int x^n \sin(ax)dx = -\frac{x^n}{a}\cos(ax)+ \frac{n}{a}\int x^{n-1}\cos(ax)dx \\
	&\int x^n \cos(ax)dx = \frac{x^n}{a}\sin(ax)- \frac{n}{a}\int x^{n-1}\sin(ax)dx		
\end{align}
\end{fancybox}

\begin{defn}[\textbf{The Wallis Cosine Formula}\index{Wallis cosine formula}]{1}
\begin{align}
\int_{0}^{\pi/2}\cos^n(x) dx = \int_{0}^{\pi/2}\sin^n(x) dx = \frac{(n-1)!!}{n!!}
\begin{cases}
\pi/2 &\textrm{ for } n=2,4,\dots\\
1 &\textrm{ for } n=3,5,\dots\\
\end{cases}
\end{align}
\end{defn}




\newpage
\section{Gaussian Integrals}
The integral of an arbitrary Gaussian function is
\begin{align}
\int x^ne^{\beta x} dx = e^{\beta x}\sum_{k=0}^{n}(-1)^k\frac{n!x^{n-k}}{(n-k)!\beta^{k+1}}+c
\end{align}
Some general Gaussian integrals evaluate as
\begin{align}
\int_{-\infty}^{\infty} e^{-\alpha x^2} dx = \sqrt{\frac{\pi}{\alpha}} 
\end{align}
\begin{multicols}{2}
	\noindent
\begin{align}
I_n&=\int x^ne^{-x/\alpha}dx \\
I_0&= -\alpha e^{-x/\alpha} \\
I_1&= -(\alpha^2+\alpha x) e^{-x/\alpha} \\
I_2&= -(2\alpha^3+2\alpha^2 x+\alpha x^2) e^{-x/\alpha} \\
I_{n+1}&=\alpha^2\frac{\partial  I_n}{\partial \alpha} 
\end{align}
\begin{align}
&\int_{0}^{\infty}e^{-x/\alpha}dx = \alpha \\
&\int_{0}^{\infty}xe^{-x/\alpha}dx = \alpha^2 \\
&\int_{0}^{\infty}x^2e^{-x/\alpha}dx = 2\alpha^3 \\
&\int_{0}^{\infty}x^ne^{-x/\alpha}dx = n!\alpha^{n+1}
\end{align}
\end{multicols}
The integral of an arbitrary Gaussian function with an n-dimensional linear term (with $n \in \mathbb{Z}$) is
\begin{align}
\int_{0}^{\infty}x^{2n}e^{-\alpha x^2}dx = \sqrt{\frac{\pi}{\alpha}}\frac{(2n-1)!!}{2^{n+1}\alpha^n} &\implies  \int_{-\infty}^{\infty}x^{2n}e^{-\alpha x^2}dx = \sqrt{\frac{\pi}{\alpha}}\frac{(2n-1)!!}{(2\alpha)^n}\\
\int_{0}^{\infty}x^{2n+1}e^{-\alpha x^2}dx = \frac{n!}{2a^{n+1}} &\implies \int_{-\infty}^{\infty}x^{2n+1}e^{-\alpha x^2}dx = 0
\end{align}
Therefore a general solution is
\begin{align}
\int_{0}^{\infty}x^ne^{-\alpha x^2}dx = 
\begin{cases}
\displaystyle
\frac{(n-1)!!}{2^{n/2+1}a^{n/2}}\sqrt{\frac{\pi}{\alpha}} & \textrm{ for $n$ even} \\
\displaystyle
\frac{[\frac{1}{2}(n-1)]!}{2a^{(n+1)/2}}& \textrm{ for $n$ odd} 
\end{cases}
\end{align}
The below form of a gaussian integral evaluates to zero when $n$ is odd due to the function being odd, but when $n$ is even, the more general integral has the following closed form
\begin{align}
\int_{-\infty}^{\infty} x^ne^{-\alpha x^2+\beta x}=\sqrt{\frac{\pi}{\alpha}}e^{\beta^2/(4\alpha)}\sum_{k=0}^{\lfloor n/2\rfloor} {{n}\choose{2k}} (2k-1)!!(2a)^{k-n}\beta^{n-2k}
\end{align}

\begin{defn}[\textbf{Saddle Point Approximation}\index{Saddle point approximation}]{1}
One can use the \textbf{saddle point approximation}\index{Saddle point approximation} to approximate integrals of a certain form (where $x_0$ is the critical point of $f'(x)=0$ - specifically the minimum)
\begin{align}
I = \int_{-\infty}^{\infty}g(x)e^{-f(x)}dx \approx g(x_0)e^{-f(x)}\sqrt{\frac{2\pi}{f''(x_0)
}}
\end{align}
\end{defn}

