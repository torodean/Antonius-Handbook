\chapter{Statistics}
\thispagestyle{fancy}
A probability distribution: Given a Poisson process, the probability of obtaining exactly m successes in n trials is given by the limit of a binomial distribution
\begin{align}
\mathcal{P}_n(m;p)={{n}\choose{m}}p^m(1-p)^{n-m}
\end{align}
Letting the sample size n become large, the distribution then approaches the Poisson Distribution
\begin{align}
\mathcal{P}(m,\lambda) &= \frac{\lambda^m}{m!}e^{-\lambda} 
\end{align}
The mean number of events is
\begin{align}
\langle m\rangle = \sum_{m=0}^{\infty}m \frac{\lambda^m}{m!}e^{-\lambda} = \lambda
\end{align}
And the standard deviation is 
\begin{align}
\sigma = \sqrt{\lambda}
\end{align}
The normal, or Gaussian distribution
\begin{align}
\mathcal{P}(x; \mu, \sigma) &= \frac{1}{\sqrt{2\pi}\sigma}exp\bigg[-\frac{(x-\mu)^2}{2\sigma^2}\bigg] \\
\mathcal{P}(a\leq x \leq b) &= \int_{a}^{b} \frac{1}{\sqrt{2\pi}\sigma}exp\bigg[-\frac{(x-\mu)^2}{2\sigma^2}\bigg] 
\end{align}
If the mean is not equal to zero, a more general distribution known as the noncentral chi-squared distribution results. In particular, if $x_i$ are independent variates with a normal distribution having means $\mu_i$ and variances $\sigma_i^2$ for $i=1, ..., n$, then 
\begin{align}
\chi^2 \equiv \sum_{i=1}^n \frac{(x_i - \mu_i)^2}{\sigma_i^2}.
\end{align}

Given some function $f(x_1, x_2, \dots, x_n)$, the error of a calculation with each respective variable being denoted by $\sigma_i$, can be determined by
\begin{align}
\sigma_f^2= \bigg( \frac{\partial f}{\partial x_1}\bigg)^2\sigma_{x_1}^2+\bigg( \frac{\partial f}{\partial x_2}\bigg)^2\sigma_{x_2}^2 + \cdots +\bigg( \frac{\partial f}{\partial x_n}\bigg)^2\sigma_{x_n}^2 = \sum_{i=1}^{n}\bigg( \frac{\partial f}{\partial x_i}\bigg)^2\sigma_{x_i}^2
\end{align}
