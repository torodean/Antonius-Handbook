\chapter{Constants and units}
\thispagestyle{fancy}
\begin{fancybox}[Physical Constants]{}	
\begin{center}
	\begin{tabular}{   l  |  c  |  l  |  l  }
		Constant & Symbol & Value & Units \\
		\hline
		Speed of light in a vacuum& c $\equiv 1/\sqrt{\mu_0\epsilon_0}$ & $2.99792458 \times 10^8$ & m/s \\
		Elementary charge& e & $1.602176565(35)\times 10^{-19}$ & C\\
		Gravitational constant& G & $6.67384(80)\times 10^{-11}$ & m$^3$kg$^{-1}$s$^{-2}$\\
		Avagadro's number& $N_a$ & $6.02214129(27)\times 10^{23}$ & mol$\cdot s^{-1}$\\
		Planck constant & $h$ & $ 6.62606872(52) \times 10^{-34}$ & J$\cdot$s \\
		& & $4.135668 \times 10^{-15}$ & eV$\cdot$s \\
		& $hc$ & 1239.84 & eV$\cdot$nm \\
		Reduced planck constant& $\hbar \equiv h/2\pi$ & $1.05\times 10^-{34}$ & J$\cdot$s\\
		Permittivity of the vacuum & $\epsilon_0$ & $8.854\times 10^{-12}$ & C$^2$N$^{-1}$m$^{-2}$ \\
<<<<<<< HEAD:Volume I -  
Mathematical Mansion/Chapters/Constants.tex
		Permeability of the vacuum & $\mu_0$ & $4\pi\times 10^{-7}$ & N/A$^{2}$ \\
=======
		Permeability of the vacuum & $\mu_0$ & $4\pi\times 10^{-7}$ & N/A$^2$ \\
>>>>>>> 3b2eac66c90fb4395f3a156a8dc0c6bbe9ec1527:Chapters/Constants.tex
		Boltzmann constant & $k_B$ & $1.38064852\times 10^{-23}$ & J/K \\
				 & & $8.61733\times 10^{-5}$ & eV/K \\
		Stefan-Boltzmann constant & $\sigma_{\textrm{B}} \equiv \frac{\pi^2k_B^4}{60\hbar^3c^3}$ & $5.670367(13)\times 10^{-8}$ & W$\cdot$m$^{-2}$K$^{-4}$ \\
		Thomson cross-section & $\sigma_e$ & $6.652\times10^{-29}$ & $m^2$ \\
		The Bohr Magneton & $\mu_B \equiv \frac{e\hbar}{2m}$ & $5.788\times 10^{-5}$ & eV/T \\
		& & $9.274\times 10^{-24}$ & Am$^2$ \\
		Mass of an electron & $m_e$ & $9.10938291(40)\times 10^{-31}$ & kg\\
		&  & $510.9989$ & keV/$c^2$\\
		Mass of a proton& $m_p$ & $1.6726218 \times 10^{-27}$ & kg\\
		&  & 938.27203 & MeV/$c^2$\\
		Mass of a neutron& $m_n$   & $1.6749274 \times 10^{-27}$ & kg \\
		& & $939.56536$ & MeV/$c^2$	\\
		Unified amu & $u$ &  $1.660538782\times 10^{-27}$ & kg \\
		  &   &  931.494028 & MeV/c$^2$ 
	\end{tabular}
\end{center}
\end{fancybox}

\begin{fancybox}[Stellar Data]{}
	\begin{center}
		\begin{tabular}{c|c|c|c|c|c}
			Spectral Type & $T_{eff}$ (K) & $M/M_\cdot$ & $L/L_\cdot$ & $R/R_\cdot$ & $V_{mag}$ \\
			\hline
			O5 & 44,500 & 60 & $7.9\times 10^5$ & 12 &-5.7 \\
			B5 & 15,400 & 5.9 & 830 & 3.9 & -1.2 \\
			A5 & 8,200 & 2.0 & 14 & 1.7 & 1.9 \\
			F5 & 6,440 & 1.4 & 3.2 & 1.3 & 3.4 \\
			G5 & 5,770 & 0.92 & 0.79 & 0.92 & 4.9 \\
			K5 & 4,350 & 0.67 & 0.15 & 0.72 & 6.7 \\
			M5 & 3,170 & 0.21 & 0.011 & 0.27 & 12.3 \\
		\end{tabular}
	\end{center}
\end{fancybox}



\newpage
\begin{fancybox}[Astronomical Constants]{1}
\begin{center}
\begin{tabular}{   l  |  c  |  l  |  l  }
Constant & Symbol & Value & Units \\
\hline
Mass of Earth& $M_\oplus$ & $5.974 \times 10^{24}$ & kg\\
Mass of Sun& $M_\odot$ &$1.989  \times 10^{30}$ & kg\\
Mass of Moon& $M_{\leftmoon}$ &$7.36 \times 10^{22}$&  kg\\
Equatorial radius of Earth& $R_\oplus$ & $6.378 \times 10^6$& m\\
Equatorial radius of Sun& $R_\odot$ &$6.6955 \times 10^8$ & m\\
Equatorial radius of Moon& $R_{\leftmoon}$ &$1.737 \times 10^{6}$ & m\\
Mean density of Earth &  & 5515  & kg$\cdot$m$^{-3}$  \\
Mean density of Sun &  & 1408  & kg$\cdot$m$^{-3}$ \\
Mean density of Moon  & & 3346  & kg$\cdot$m$^{-3}$ \\
Earth-Moon distance& &$3.84 \times 10^8$ & m\\
Earth-Sun distance& &$1.496 \times 10^{11}$ & m \\
Luminosity of Sun & $L_\odot$  & $3.839\times 10^{26}$  & W  \\
Effective temp. of Sun &   & 5778  & K  \\
Hubble constant & $H_0$  & $70\pm 5$  & km$\cdot$s$^{-1}$Mpc$^{-1}$  \\
Parsec& pc & 206264.81 & AU\\
&  & $3.0856776 \times 10^{16}$ & m\\
& & $3.2615638$ & ly \\
Astronomical Unit & AU & $1.496 \times 10^{11}$ & m \\
Light year& ly & $9.461 \times 10^{15}$ & m \\
1 year on Earth& yr & 365.25 & days \\
&  & $3.15576 \times 10^{7}$& s 
\end{tabular}
\end{center}
\end{fancybox}

\begin{fancybox}[Solar System]{}
	\begin{center}
		\begin{tabular}{   l  |  c  |  l  |  l  | l}
			Planet & Symbol & Mass (kg) & Radius (m) & Sun-Distance (km) \\
			\hline
			Mercury & \mercury & $3.285 \times 10^{23}$ & 2.44 $\times 10^{6}$ & $5.791 \times 10^{10}$ \\
			Venus & \venus & $4.867 \times 10^{24}$ & $6.052 \times 10^{6}$ & $1.082 \times 10^{11}$   \\
			Mars & \mars & $6.39 \times 10^{23}$ & $3.390 \times 10^{6}$ & $2.279 \times 10^{11}$ \\
			Jupiter& \jupiter &$1.898 \times 10^{27}$ & $3.83 \times 10^{11}$ & $7.785 \times 10^{11}$  \\
			Saturn & \saturn & $5.683 \times 10^{26}$ & $5.8232 \times 10^{7}$ & $1.429 \times 10^{12}$  \\
			Uranus & \uranus & $8.681 \times 10^{25}$ & $2.5362 \times 10^{7}$ & $2.871 \times 10^{12}$  \\
			Neptune & \neptune & $1.024 \times 10^{26}$ & $2.4622 \times 10^{7}$ & $4.498 \times 10^{12}$  \\
			Pluto & \pluto & $1.309 \times 10^{22}$ & $1.187 \times 10^6$ & $5.906 \times 10^{12}$ 
		\end{tabular}
	\end{center}
\end{fancybox}

\newpage
\begin{fancybox}[Unit conversions]{}
	The International System of Units (SI) defines seven units of measure as a basic set from which all other SI units can be derived. These are [length](m), [time](s), [mass](kg), [electric current] $\equiv$ [Ampere](A), [temperature](K), [luminous intensity](cd), [amount of substance](mol).
	\begin{center}
		\begin{tabular}{c|l|l}
			Unit Symbol & Unit & Equivalence \\
			\hline
			C & [Coulomb] & [Ampere][time] \\
			N & [Newton] & [mass][length][time]$^{-2}$ \\
			P & [Pascal]  & [mass][length]$^{-1}$[time]$^{-2}$ \\
			J & [Joule]  & [mass][length]$^{2}$[time]$^{-2}$ \\
			W & [Watt]  & [mass][length]$^{2}$[time]$^{-3}$ \\
			 & & [Ohm][Ampere]$^2$ \\
			 & & [Volt]$^2$[Ohm]$^{-1}$ \\
			V & [Volt]  & [mass][length]$^{2}$[time]$^{-3}$[Ampere]$^{-1}$ \\
			Wb & [Weber]  & [mass][length]$^{2}$[time]$^{-2}$[Ampere]$^{-1}$ \\
			T & [Tesla]  & [mass][time]$^{-2}$[Ampere]$^{-1}$ \\
			H & [henry]  & [mass][length]$^{2}$[time]$^{-2}$[Ampere]$^{-2}$ \\
			$\Omega$ & [Ohm]  & [mass][length]$^{2}$[time]$^{-3}$[Ampere]$^{-2}$ \\
			F & [Farad]  & [mass]$^{-1}$[length]$^{-2}$[time]$^{4}$[Ampere]$^{2}$ \\
			Hz & [Hertz]  & [time]$^{-1}$ 
		\end{tabular}
	\end{center}
\end{fancybox}


\begin{fancybox}[Number Sets ($i \equiv \sqrt{-1}$)]{}
	\begin{center}
		\begin{tabular}{c|l||c|l}
			Symbol &  Set  & Symbol & Set   \\
			\hline
			$\mathbb{R}$ & Real numbers & $\emptyset$ & \{\} \\
			$\mathbb{N}\equiv \mathbb{N}_1$ & \{1,2,3,4,\dots\} & $\mathbb{Z}$ & \{\dots,-2,1,0,1,2,\dots\} \\
			$\mathbb{Z}^+ \equiv \mathbb{N}_0$ & \{0,1,2,3,\dots\} & $\mathbb{Z}^-$ & \{0,-1,-2,-3,-4,\dots\} \\
			$\mathbb{C}$ & $\{x+iy | x,y \in \mathbb{R}\}$ & $\mathbb{Q}$ & $\{\frac{x}{y} | x,y \in \mathbb{Z}\}$ \\
			$\mathbb{I}$ & $\{ix|x\in \mathbb{R}\}$ & $\mathbb{U}$ & Universal Set \footnote{Definition: The set containing all objects or elements and of which all other sets are subsets.} \\
			$\mathbb{A}$ & Algebraic Numbers\footnote{Any number that is a solution to a polynomial equation with rational coefficients.} & $\mathbb{T}$ & Transcendental Numbers \footnote{Any number that is not an Algebraic Number.} 
		\end{tabular}
	\end{center}
\end{fancybox}

