\chapter{Mathematical Analysis}
\thispagestyle{fancy}

\begin{defn}[Field\index{Field}]{1}
	A \textit{field} $F$ can be defined as a commutative ring with identity such that for each $a\in F$ where $a \neq 0$, there is an element $a^{-1} \in F$ such that $a a^{-1}=1$.
\end{defn}

\begin{theo}[Rational Zeros Theorem\index{Rational Zeros Theorem}\cite{Elementary Analysis}]{1}
	Suppose $c_0, c_1, \dots,c_n$ are integers and r is a rational number satisfying the polynomial equation
	\begin{align}
		c_nx^n+c_{n-1}x^{n-1}+\cdots + c_1x+c_0=0 \label{RZT}
	\end{align}
	where $n\geq 1, c_n \neq 0$ and $c_0 \neq 0$. Let $r=\frac{c}{d}$ where $c,d$ are integers having no common factors and $d \neq 0$. Then $c$ divides $c_0$ and $d$ divides $c_n$.
	
	\hspace{1cm} In other words, the only rational candidates for solutions to (\ref{RZT}) have the form $\frac{c}{d}$ where $c$ divides $c_0$ and $d$ divides $c_n$.
\end{theo}

\begin{theo}[Properties of Absolute Values in $\mathbb{R}$]{1}
	For all $a,b \in \mathbb{R}$, $|a| \geq 0$, $|ab| = |a|\cdot|b|$, and $|a+b| \leq |a|+|b|$ (The \textit{triangle inequality}\index{Triangle inequality}).
\end{theo}

\begin{defn}[Minimum\index{Minimum} and Maximum\index{Maximum} \cite{Elementary Analysis}]{1}
	Let $S$ be a non-empty subset of $\mathbb{R}$.
	\begin{enumerate}
		\item[(a)] If $S$ contains a largest element $s_m$ [that is $s_m \in S$ and $s \leq s_m$ for all $s \in S$], then we call $s_m$ the \textit{maximum} of $S$ and write $s_m = \max S$.
		
		\item[(b)] If $S$ contains a smallest element $s_m$ [that is $s_m \in S$ and $s_m \leq s$ for all $s \in S$], then we call $s_m$ the \textit{minimum} of $S$ and write $s_m = \min S$.
	\end{enumerate}
\end{defn}

\begin{defn}[Bounded Sets \cite{Elementary Analysis}]{1}
	Let $S$ be a non-empty subset of $\mathbb{R}$.
	\begin{enumerate}
		\item[(a)] If $M \in \mathbb{R}$ satisfies $s \leq M$ for all $s \in S$, then $M$ is
		called an \textit{upper bound} of $S$ and the set $S$ is said to be bounded above.
		
		\item[(b)] If $m \in \mathbb{R}$ satisfies $m \leq s$ for all $s \in S$, then $m$ is
		called a \textit{lower bound} of $S$ and the set $S$ is said to be bounded below.
		
		\item[(c)] The set $S$ is said to be bounded if it is bounded above and bounded below. Thus $S$ is bounded if there exist $m,M \in \mathbb{R}$ such that $S \subseteq [m, M]$
	\end{enumerate}
\end{defn}

\newpage

\begin{defn}[Supremum\index{Supremum} and Infimum\index{Infimum} \cite{Elementary Analysis}]{1}
	Let $S$ be a non-empty subset of $\mathbb{R}$.
	\begin{enumerate}
		\item[(a)] If $S$ is bounded above and $S$ has a least upper bound, then we will call it the \textit{supremum} of $S$ and denote it $\sup S$.
		
		\item[(b)] If $S$ is bounded below and $S$ has a greatest lower bound, then we will call it the \textit{infimum} of $S$ and denote it $\inf S$.
	\end{enumerate}
\end{defn}

\begin{defn}[Completeness Axiom\index{Completeness Axiom} \cite{Elementary Analysis}]{1}
	Every nonempty subset $S$ of $\mathbb{R}$ that is bounded above has a least upper bound. In other words, $\sup S$ exists and is a real number.
\end{defn}

\begin{defn}[Archimedean Property\index{Archimedean property} \cite{Elementary Analysis}]{1}
	If $a>0$ and $b>0$, then for some positive integer $n$, we have $na >b$.
\end{defn}

\begin{defn}[Denseness of $\mathbb{Q}$\index{Denseness of $\mathbb{Q}$} \cite{Elementary Analysis}]{1}
	If $a,b \in \mathbb{R}$ and $a < b$, then there is a rational $r \in \mathbb{Q}$ such that $a < r < b$.
\end{defn}

\section{Sequences and Series}

\begin{defn}[A Sequence\index{Sequence}]{1}
	A sequence is a function $\{n\in\mathbb{Z}:n\geq m\} \rightarrow \mathbb{R},n\mapsto s_n$.
	\begin{enumerate}
		\item $m=0$ or $m=1$ most often.
		\item Often denoted as $(s_1,s_2,s_3,\dots)$, $(s_n)_{n=m}^\infty$, $(s_n)_{n\in\mathbb{N}}$, or $(s_n)$.		
	\end{enumerate}
\end{defn}

\begin{defn}[Limits\index{Limit}, Converging\index{Converge}, and Diverging\index{Diverge} \cite{Elementary Analysis}]{1}
	A sequence $(s_n)$ of real numbers is said to \textit{converge} to the real number $s$ provided that for each $\epsilon > 0$ there exists a number $N$	such that $n > N \implies |s_n-s|<\epsilon$. If $(s_n)$ converges to $s$, we will write $\lim_{n\rightarrow\infty} s_n = s$, or $s_n \rightarrow s$. The	number $s$ is called the \textit{limit} of the sequence $(s_n)$. A sequence that does not converge to some real number is said to \textit{diverge}.
\end{defn}

\newpage

\begin{theo}[Sequence and Limit Properties\index{Limit}\index{Sequence} \cite{Elementary Analysis}]{1}
	\begin{enumerate}[i]
		\item Convergent sequences are bounded.
		\item If the sequence $(s_n)$ converges to $s$ and $k\in\mathbb{R}$, then the sequence $(ks_n)$ converges to $ks$. That is, $\lim ks_n=k\cdot \lim s_n$.
		\item If the sequence $(s_n)$ converges to $s$ and $(t_n)$ converges to $t$, then $(s_n+t_n)$ converges to $s+t$. That is, $\lim s_n+t_n=\lim s_n+\lim t_n$.
		\item If the sequence $(s_n)$ converges to $s$ and $(t_n)$ converges to $t$, then $(s_n t_n)$ converges to $st$. That is, $\lim s_n\cdot t_n=(\lim s_n)(\lim t_n)$.
		\item Suppose $(s_n)$ converges to $s$ and $(t_n)$ converges to $t$. If $s \neq 0$ and
		$s_n \neq 0$ for all $n$, then $(\frac{t_n}{s_n})$ converges to $\frac{t}{s}$.		
		\item Let $(s_n)$ and $(t_n)$ be sequences such that $\lim s_n=\pm\infty$ and $\lim t_n > 0$. Then $\lim s_n t_n = \pm\infty$.
		\item For a sequence $(s_n)$ of positive real numbers, we have $\lim s_n = \infty$
		if and only if $\lim(\frac{1}{s_n}) = 0$.
		\item All bounded monotonic sequences converge.
		\item If $(s_n)$ is an unbounded increasing or decreasing sequence, then $\lim s_n = \pm\infty$ respectively.
		\item If $t \in \mathbb{R}$, then there is a subsequence of $(s_n)$ converging to $t$ if and only if the set $\{n \in \mathbb{N} : |s_n − t| < \epsilon\}$ is infinite for all $\epsilon > 0$.
		\item If the sequence $(s_n)$ is unbounded above or below, it has a subsequence with limit $\pm \infty$ respectively.
		\item Let $(s_n)$ be any sequence. There exists a monotonic subsequence	whose limit is $\lim \sup s_n$, and there exists a monotonic subsequence	whose limit is $\lim \inf s_n$.
	\end{enumerate}
\end{theo}


\begin{defn}[Divergence\index{Divergence} \cite{Elementary Analysis}]{1}
	Let $(s_n)$ be a sequence.
	\begin{enumerate}[i]
		\item We write $\lim s_n = \infty$ (diverges to $+\infty$) provided for each $M>0$, there is a number $N$ such that $n>N \implies s_n>M$.
		\item We write $\lim s_n = -\infty$ (diverges to $-\infty$) provided for each $M<0$, there is a number $N$ such that $n>N \implies s_n<M$.
	\end{enumerate}	
\end{defn}

\begin{defn}[Monotonic Sequences\index{Monotonic Sequences} \cite{Elementary Analysis}]{1}
	A sequence $(s_n)$ of real numbers is called an \textit{increasing sequence}	if $s_n \leq s_{n+1}$ for all $n$, and $(s_n)$ is called a \textit{decreasing sequence} if
	$s_n \geq s_{n+1}$ for all $n$. Note that if $(s_n)$ is increasing, then $s_n \leq s_m$
	whenever $n < m$. A sequence that is increasing or decreasing will	be called a \textit{monotone sequence} or a \textit{monotonic sequence}.
	\begin{enumerate}[(i)]
		\item Monotonic sequences will always either converge, or diverge to $\pm \infty$.
	\end{enumerate}
\end{defn}

\newpage

\begin{theo}[Limits of Supremum\index{Supremum} and Infimum\index{Infimum} \cite{Elementary Analysis}]{1}
	Let $(s_n)$ be a sequence in $\mathbb{R}$.
	\begin{enumerate}[(i	)]
		\item If $\lim s_n$ is defined (as a real number or $\pm\infty$), then $\lim \inf s_n = \lim s_n = \lim \sup s_n$.
		\item If $\lim \inf s_n = \lim \sup s_n$, then $\lim s_n$ is defined and $\lim s_n =
		\lim \inf s_n = \lim \sup s_n$.
		\item If $(s_n)$ converges to a positive real number $s$ and $(t_n)$ is any sequence, then $\lim \sup s_n t_n = s\cdot \lim \sup t_n$.
		\item If $(s_n)$ is any sequence of nonzero real numbers then
		\begin{align}
			\lim \inf \left|\frac{s_{n+1}}{s_n}\right| \leq \lim \inf |s_n|^{1/n} \leq \lim \sup |s_n|^{1/n} \leq \lim \sup \left|\frac{s_{n+1}}{s_n}\right| \nonumber
		\end{align}
	\end{enumerate}
\end{theo}

\begin{defn}[Cauchy Sequence\index{Cauchy Sequence} \cite{Elementary Analysis}]{1}
	A sequence $(s_n)$ of real numbers is called a \textit{Cauchy sequence} if for each $\epsilon > 0$ there exists a number $N$ such that $m,n > N \implies |s_n-s_m|<\epsilon$.
	\begin{enumerate}[(i)]
		\item Cauchy sequences are bounded.
		\item A sequence is a convergent sequence if and only if it is a Cauchy sequence. 
	\end{enumerate}	
\end{defn}

\begin{defn}[Subsequence\index{Subsequence} \cite{Elementary Analysis}]{1}
	Suppose $(s_n)_{n\in N}$ is a sequence. A \textit{subsequence} of this sequence is a sequence of the form $(t_k)_{k\in N}$ where for each $k$ there is a positive	integer $n_k$ such that	$n_1 < n_2 < \cdots < n_k < n_{k+1} < \cdots$ and $t_k = s_{n_k}$.
	\begin{enumerate}[(i)]
		\item If the sequence $(s_n)$ converges, then every subsequence converges to the same limit.
		\item Every sequence $(s_n)$ has a monotonic subsequence.
		\item Every bounded sequence has a convergent subsequence (Bolzano-Weierstrass Theorem \index{Bolzano-Weierstrass Theorem}).
	\end{enumerate}
\end{defn}


\begin{defn}[Subsequential limit\index{Subsequential limit} \cite{Elementary Analysis}]{1}
	Let $(s_n)$ be a sequence in $\mathbb{R}$. A \textit{subsequential limit} (occasionally denoted $S$) is any real number or symbol $\pm \infty$ that is the limit of some subsequence of $(s_n)$.
	\begin{enumerate}[(i)]
		\item $S$ is non-empty.
		\item $\sup S = \lim \sup s_n$ and $\inf S = \lim \inf s_n$.
		\item $\lim s_n$ exists if and only if $S$ has exactly one element, namely	$\lim s_n$.
		\item Suppose $(t_n)$ is a sequence in $S\cap R$ and that $t = \lim t_n$. Then $t$ belongs to $S$.
	\end{enumerate}
\end{defn}

\begin{defn}[Cauchy Critereon\index{Cauchy Critereon} \cite{Elementary Analysis}]{1}
	We say a series $\sum a_n$ satisfies the \textit{Cauchy Critereon} if its subsequence ($s_n$) of partial sums is a Cauchy sequence: for each $\epsilon > 0$ there exists a number $N$ such that $m,n >N$ implies $|s_n-s_m|<\epsilon$. 
	\begin{enumerate}
		\item A series converges if and only if it satisfies the Cauchy criterion.
	\end{enumerate}
\end{defn}

\section{Convergence/Divergence Tests}

\begin{fancybox}[Comparison Test\index{Comparison Test} \cite{Elementary Analysis}]{1}
	Let $\sum a_n$ be a series where $a_n \geq 0$ for all $n$.
	\begin{enumerate}[(i)]
		\item If $\sum a_n$ converges and $|b_n| \leq a_n$ for all $n$, then $\sum b_n$ converges.
		\item If $\sum a_n = \infty$ and $b_n \geq a_n$ for all $n$, then $\sum b_n = \infty$. 
	\end{enumerate}
\end{fancybox}

\begin{fancybox}[Ratio Test\index{Ratio Test} \cite{Elementary Analysis}]{1}
	Let $\sum a_n$ be a series of nonzero terms.
	\begin{enumerate}[(i)]
		\item $\sum a_n$ converges absolutely if $\lim \sup \left| \frac{a_{n+1}}{a_n}\right| <1$.
		\item $\sum a_n$ diverges if $\lim \inf \left| \frac{a_{n+1}}{a_n}\right| >1$.
		\item Otherwise, $\lim \inf \left| \frac{a_{n+1}}{a_n}\right| \leq 1 \leq \lim \sup \left| \frac{a_{n+1}}{a_n}\right|$ and the test gives no information.
	\end{enumerate}
\end{fancybox}

\begin{fancybox}[Root Test\index{Root Test} \cite{Elementary Analysis}]{1}
	Let $\sum a_n$ be a series and let $\alpha = \lim \sum |a_n|^{1/n}$.
	\begin{enumerate}[(i)]
		\item The series $\sum a_n$ converges absolutely if $\alpha <1$.
		\item The series $\sum a_n$ diverges if $\alpha >1$.
	\end{enumerate}
\end{fancybox}


\begin{fancybox}[Integral Test\index{Integral Test} \cite{Elementary Analysis}]{1}
	\begin{enumerate}[(i)]
		\item If $\lim\limits_{n \rightarrow \infty} \int_{0}^{n} f(x) dx =\infty$ then the series $\sum f(x)$ will diverge.
		\item If $\lim\limits_{n \rightarrow \infty} \int_{0}^{n} f(x) dx <\infty$ then the series $\sum f(x)$ will converge.
	\end{enumerate}
\end{fancybox}

\newpage

\begin{theo}[Miscellaneous Convergent Properties and Theorems \cite{Elementary Analysis}]{1}
	\begin{enumerate}[(i)]
		\item If a series $\sum a_n$ converges, then $\lim a_n = 0$. 
		\item $\sum \frac{1}{n^p}$ converges if and only if $p > 1$. 
	\end{enumerate}
\end{theo}



\begin{theo}[Alternating Series Theorem\index{Alternating Series} \cite{Elementary Analysis}]{1}
	If $a_1 \geq a_2 \geq \cdots \geq a_n \geq \cdots \geq 0$ and $\lim a_n =0$, then the alternating series $\sum(-1)^{n+1}a_n$ converges. Moreover, the partial sums $s_n = \sum_{k=1}^{n}(-1)^{k+1}a_k$ satisfy $|s-s_n|\leq a_n$ for all $n$.
\end{theo}

\section{Functions}

\begin{defn}[Continuous Function\index{Continuous function} \cite{Elementary Analysis}]{1}
	Let $f$ be a real-valued function whose domain is a subset of $\mathbb{R}$. The function $f$ is \textit{continuous} at $x_0$ in dom($f$) if, for every sequence ($x_n$)	in dom($f$) converging to $x_0$, we have $\lim_n f(x_n) = f(x_0)$. If $f$ is continuous at each point of a set $S \subseteq$ dom($f$), then $f$ is said to	be \textit{continuous} on $S$. The function $f$ is said to be \textit{continuous} if it is continuous on dom($f$).
\end{defn}

\begin{theo}[Continuous Function\cite{Elementary Analysis}]{1}
	Let $f$ be a real-valued function with dom($f$) $\subseteq \mathbb{R}$. Then
	$f$ is \textit{continuous} at $x_0$ in dom($f$) if and only if for each $\epsilon > 0$, there exists $\delta > 0$ such that $x\in$ dom($f$) and $|x-x_0|<\delta$ imply $|f(x)-f(x_0)|<\epsilon$.
\end{theo}

\begin{defn}[Continuous Functions\index{Continuous functions} \cite{Elementary Analysis}]{1}
	Let $f$ be a real-valued function whose domain is a subset of $\mathbb{R}$.
	\begin{enumerate}[(i)]
		\item $f$ is continuous at $x_0$ in dom($f$) if, for every sequence ($x_n$)
		in dom($f$) converging to $x_0$ , we have $\lim_n f (x_n) = f(x_0)$. If $f$ is
		continuous at each point of a set $S \subseteq $ dom($f$), then $f$ is said to
		be continuous on $S$. The function $f$ is said to be continuous if it is continuous on dom($f$).
		\item $f$ is continuous at $x_0$ in dom($f$) if and only if for each $\epsilon > 0$ there exists $\delta >0$ such that $x\in$ dom($f$) and $|x-x_0|<\delta$ imply $|f(x)-f(x_0)|<\epsilon$.
		\item If $f$ is continuous
		at $x_0$ in dom($f$), then $|f|$ and $kf , k \in\mathbb{R}$, are continuous at $x_0$.
	\end{enumerate}
\end{defn}

\begin{theo}[\cite{Elementary Analysis}]{1}
	If $f$ is continuous at $x_0$ and $g$ is continuous at $f(x_0)$, then $g \circ f$ is continuous at $x_0$.
\end{theo}

\newpage

\begin{theo}[\cite{Elementary Analysis}]{1}
	Let $f$ and $g$ be real-valued functions that are continuous at $x_0$ in $\mathbb{R}$. Then
	\begin{enumerate}[(i)]
		\item $f+g$ is continuous at $x_0$;
		\item $fg$ is continuous at $x_0$;
		\item $\frac{f}{g}$ is continuous at $x_0$ if $g(x_0) \neq 0$.
	\end{enumerate}
\end{theo}

\begin{theo}[\cite{Elementary Analysis}]{1}
	Let $f$ be a continuous real-valued function on a closed interval $[a, b]$.
	Then $f$ is a bounded function. Moreover, $f$ assumes its maximum and minimum values on $[a, b]$; that is, there exist $x_0, y_0 \in [a, b]$ such that $f(x_0) \leq f(x) \leq f(y_0)$ for all $x \in [a, b]$.
\end{theo}

\begin{theo}[Intermediate Value Theorem\index{Intermediate Value Theorem}\cite{Elementary Analysis}]{1}
	If $f$ is a continuous real-valued function on an interval $I$, then $f$ has
	the intermediate value property on $I$: Whenever $a, b \in I, a < b$ and $y$
	lies between $f(a)$ and $f(b)$ [i.e., $f(a) < y < f(b)$ or $f(b) < y < f(a)$],
	there exists at least one $x \in (a, b)$ such that $f(x) = y$.
	\begin{enumerate}[(i)]
		\item \textit{Corollary.} If $f$ is a continuous real-valued function on an interval $I$, then the set $f(I) = \{f(x) : x \in I\}$ is also an interval or a single point.
	\end{enumerate}
\end{theo}

\begin{theo}[\cite{Elementary Analysis}]{1}
	Let $f$ be a continuous strictly increasing function on some interval $I$. Then $f(I)$ is an interval $J$ by the Intermediate Value Theorem and $f^{−1}$ represents a function with domain $J$. The function $f^{−1}$ is a continuous strictly increasing function on $J$.
\end{theo}

\begin{theo}[\cite{Elementary Analysis}]{1}
	Let $g$ be a strictly increasing function on an interval $J$ such that $g(J)$ is an interval $I$. Then $g$ is continuous on $J$.
\end{theo}

\begin{theo}[\cite{Elementary Analysis}]{1}
	Let $f$ be a one-to-one continuous function on an interval $I$. Then
	$f$ is strictly increasing $[x_1 < x_2 \implies f(x_1) < f(x_2)]$ or strictly
	decreasing $[x_1 < x_2 \implies f(x_1) > f(x_2)]$.
\end{theo}

\newpage

\begin{defn}[Uniformly Continuous\index{Uniformly Continuous}\cite{Elementary Analysis}]{1}
	Let $f$ be a real-valued function defined on a set $S \subseteq \mathbb{R}$. Then $f$ is \textit{Uniformly Continuous} on $S$ if for each $\epsilon > 0$ there exists $\delta > 0$ such that $x, y \in S$ and $|x-y| < \delta$ imply $|f(x)- f(y)| < \epsilon$. We will say $f$ is uniformly continuous if $f$ is \textit{uniformly continuous} on dom($f$).
	\begin{enumerate}
		\item If $f$ is continuous on a closed interval $[a, b]$, then $f$ is uniformly continuous on $[a, b]$.
		\item If $f$ is uniformly continuous on a set $S$ and ($s_n$) is a Cauchy sequence in $S$, then ($f(s_n)$) is a Cauchy sequence.
		\item A real-valued function $f$ on $(a, b)$ is uniformly continuous on $(a, b)$ if and only if it can be extended to a continuous function $\bar{f}$ on $[a, b]$.
	\end{enumerate}
\end{defn}



