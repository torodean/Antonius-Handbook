\chapter{Mathematical Analysis}
\thispagestyle{fancy}

\begin{defn}[Field]{1}
	A \textit{field} $F$ can be defined as a commutative ring with identity such that for each $a\in F$ where $a \neq 0$, there is an element $a^{-1} \in F$ such that $a a^{-1}=1$.
\end{defn}

\begin{theo}[Rational Zeros Theorem\index{Rational Zeros Theorem}\cite{Elementary Analysis}]{1}
	Suppose $c_0, c_1, \dots,c_n$ are integers and r is a rational number satisfying the polynomial equation
	\begin{align}
		c_nx^n+c_{n-1}x^{n-1}+\cdots + c_1x+c_0=0 \label{RZT}
	\end{align}
	where $n\geq 1, c_n \neq 0$ and $c_0 \neq 0$. Let $r=\frac{c}{d}$ where $c,d$ are integers having no common factors and $d \neq 0$. Then $c$ divides $c_0$ and $d$ divides $c_n$.
	
	\hspace{1cm} In other words, the only rational candidates for solutions to (\ref{RZT}) have the form $\frac{c}{d}$ where $c$ divides $c_0$ and $d$ divides $c_n$.
\end{theo}

\begin{theo}[Properties of Absolute Values in $\mathbb{R}$]{1}
	For all $a,b \in \mathbb{R}$, $|a| \geq 0$, $|ab| = |a|\cdot|b|$, and $|a+b| \leq |a|+|b|$.
\end{theo}

\begin{defn}[Minimum\index{Minimum} and Maximum\index{Maximum} \cite{Elementary Analysis}]{1}
	Let $S$ be a non-empty subset of $\mathbb{R}$.
	\begin{enumerate}
		\item[(a)] If $S$ contains a largest element $s_m$ [that is $s_m \in S$ and $s \leq s_m$ for all $s \in S$], then we call $s_m$ the \textit{maximum} of $S$ and write $s_m = \max S$.
		
		\item[(b)] If $S$ contains a smallest element $s_m$ [that is $s_m \in S$ and $s_m \leq s$ for all $s \in S$], then we call $s_m$ the \textit{minimum} of $S$ and write $s_m = \min S$.
	\end{enumerate}
\end{defn}

\begin{defn}[Bounded Sets \cite{Elementary Analysis}]{1}
	Let $S$ be a non-empty subset of $\mathbb{R}$.
	\begin{enumerate}
		\item[(a)] If $M \in \mathbb{R}$ satisfies $s \leq M$ for all $s \in S$, then $M$ is
		called an \textit{upper bound} of $S$ and the set $S$ is said to be bounded above.
		
		\item[(b)] If $m \in \mathbb{R}$ satisfies $m \leq s$ for all $s \in S$, then $m$ is
		called a \textit{lower bound} of $S$ and the set $S$ is said to be bounded below.
		
		\item[(c)] The set $S$ is said to be bounded if it is bounded above and bounded below. Thus $S$ is bounded if there exist $m,M \in \mathbb{R}$ such that $S \subseteq [m, M]$
	\end{enumerate}
\end{defn}

