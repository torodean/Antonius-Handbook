\chapter{Thermodynamics}
\thispagestyle{fancy}
\begin{multicols}{2}
Useful constants: the specific heat of water is $c$
\begin{align}
c &\textrm{ = 4186 J/(kg$\cdot$K)}. \\
1 \textrm{ cal} &= 4.186 \textrm{ J}
\end{align}
Temperature relationships.
\begin{align}
^\circ F &= \frac{9}{5}^\circ C + 32 \\
 ^\circ C &= \frac{5}{9}(^\circ F - 32) \\
 ^\circ K &= ^\circ C + 273.15
\end{align}
The heat required to raise the temperature of a mass $m$ by $\Delta T$ is
\begin{align}
Q=cm\Delta T
\end{align}he temperature of an object determines the radiated power of the object, which is given by the \textbf{Stefan-Boltzmann equation}\index{Stefan-Boltzmann equation}
\begin{align}
P_{radiated} &=\sigma \epsilon A T^4 \\
\sigma &= 5.67 \times 10^{-8} \textrm{ W/$K^4m^2$} \\
\epsilon &= \textrm{emissivity, and }0 \leq \epsilon \leq 1
\end{align}
The work done on a system in going from initial volume ($V_i$) to a final volume ($V_f$) is
\begin{align}
W=\int dW = \int_{V_i}^{V_f} pdV.
\end{align}
The first law of thermodynamics, with internal energy ($dU$), heat transferred ($dQ$), pressure ($P$) and volume ($dV$)
\begin{align}
\Delta E_{internal}&=Q-W \\
dU &= dQ - PdV
\end{align}

different processes include
\begin{enumerate}[(i)]
	\item An adiabatic process is one where $dQ=0$.
	\item In a constant-volume process, $W=0$.
	\item In a closed-loop process, $Q=W$.
	\item In an adiabatic free expansion, $Q=W=\Delta E_{internal}=0$.
\end{enumerate}
The efficiency of a system is defined by
\begin{align}
Eff = \frac{W_{cycle}}{Q_{in}}=1-\frac{Q_c}{Q_h}<100\%
\end{align}
If heat is added to an object, its change in temperature (with $C=$heat capacity of the object) is given by 
\begin{align}
\Delta T &= \frac{Q}{C}
\end{align}
If heat is added to an object with mass m, its change in temperature (with $c=$specific heat of the object) is given by 
\begin{align}
\Delta T &= \frac{Q}{cm}
\end{align}
The ideal gas law\index{Ideal gas law}: For an ideal gas of n particles in a volume V at pressure and temperature P and T, the equation of state is
\begin{align}
PV &=nN_AkT\equiv nRT
\end{align}
With a constant number of moles we get from the ideal gas law the following relation:
\begin{align}
\frac{P_1V_1}{T_1}=\frac{P_2V_2}{T_2}
\end{align}
\textbf{Dalton's law}\index{Dalton's law} - The total pressure exerted by a mixture of gases is equal to the sum of the partial pressures pf the gases in the mixture.
\begin{align}
P_{total}=P_1+P_2+P_3+\dots + P_n
\end{align}
The work done by an ideal gas at constant temperature is
\begin{align}
W=nRT\ln\bigg(\frac{V_f}{V_i}\bigg)
\end{align}
The average kinetic energy of an ideal gas
\begin{align}
K_{ave} &=\frac{1}{N}\sum_{i=1}^{N}K_i\\&=\frac{1}{N}\sum_{i=1}^{N}\frac{1}{2}mv_i^2\\&= \frac{1}{2}mv_{rms}^2
\end{align}
The root-mean-square speed of gas molecules is
\begin{align}
v_{rms}=\sqrt{\frac{1}{N}\sum_{i=1}^{N}v_i^2}=\sqrt{\frac{3RT}{m}}
\end{align}
For an adiabatic process (with $C_V$=specific heat at constant volume, $C_P$=specific heat at constant pressure), we have
\begin{align}
dE_{internal} &=-PdV=nC_VdT \\
PV^\gamma &= \textrm{constant} \\
\gamma &= \frac{C_P}{C_V} \\
P_fV_f^\gamma &= P_iV_i^\gamma \\
T_fV_f^{\gamma-1} &= T_iV_i^{\gamma-1}
\end{align}
In classical thermodynamics the entropy S is defined by
The fundamental temperature $\tau$ is defined by the relation
\begin{align}
	\frac{1}{T} = \left(\frac{\partial S}{\partial U}\right)_N.
\end{align}
\end{multicols}
