\chapter{Thermal \& Statistical Physics}
\thispagestyle{fancy}

\begin{multicols}{2}
	\section{States of a Model System}
	The multiplicity function\index{Multiplicity function} for a system of N magnets with a spin excess $2s=	N_{\uparrow}-N_{\downarrow}$ is
	\begin{align}
		g(N,s)=\frac{N!}{(\frac{N}{2}+s)!(\frac{N}{2}-s)!} = \frac{N!}{N_{\uparrow}!N_{\downarrow}!}.
	\end{align}
	It is often useful to evaluate $g(N,s)$ within a logarithm in which the \textbf{Stirling approximation}\index{Stirling approximation} becomes useful.
	\begin{align}
		N! \approx N^N\sqrt{2\pi N} \exp\left(-N+\frac{1}{12N}+\cdots\right).
	\end{align}
	It is often useful to take the logarithm of this which gives
	\begin{align}
		\log N!\cong \frac{\log 2\pi}{2}+\left(N+\frac{1}{2}\right)
		\log N-N.
	\end{align}
	In the limit $s/N << 1$, with $N>>1$, we have the Gaussian approximation 
	\begin{align}
		g(N,s) &\cong g(N,0)\exp\left(\frac{-2s^2}{N}\right) \\
		g(N,0)&\simeq 2^N\sqrt{\frac{2}{\pi N}}.
	\end{align}
	The exact value of $g(N,0)$ is given by
	\begin{align}
		g(N,0) = \frac{N!}{(N/2)!(N/2)!}.
	\end{align}
	The average value, or mean value, of a function f(s) taken over a probability distribution P(s) is defined as
	\begin{align}
		\langle f \rangle &=\sum_{s} f(s)P(s), \\
		1 &= \sum_{s} P(s).
	\end{align}
	The binomial distribution has the property 
	\begin{align}
		\sum_{s}g(N,s)=2^N.
	\end{align}
	If all states of the model spin system are equally likely, the average value of $s^2$ is
	\begin{align}
		\langle s^2 \rangle = \frac{\int_{-\infty}^{\infty}s^2 g(N,s) ds }{\int_{-\infty}^{\infty} g(N,s) ds } = \frac{N}{4}
	\end{align}
	The energy interaction of a single magnetic moment\index{Magnetic!Moment} $\vec{m}$ with a fixed external magnetic field $\vec{B}$ is
	\begin{align}
		U=-\vec{m}\cdot \vec{B}.
	\end{align}
	For a model system of N elementary magnets, each with two allowed orientations in a uniform magnetic field $\vec{B}$, the total potential energy U is
	\begin{align}
		U &=\sum_{i=0}^{N}U_i=-\vec{B}\cdot \sum_{i=0}^{N}m_i \\
		&=-2smB = -MB.
	\end{align}
	\section{Entropy And Temperature}
	If $P(s)$ is the probability that a system is in the state $X$, the average value of a quantity $X$ is 
	\begin{align}
		\langle X \rangle = \sum_{s}X(s)P(s).
	\end{align}
	The number of combined systems 1 and 2 (with $s=s_1+s_2$) is
	\begin{align}
		g(s) = \sum_s g_1(s_1)g_2(s-s_1).
	\end{align}
	The relation $s=k_B\sigma$ connects the conventional entropy S with the fundamental entropy $\sigma$. The \textbf{entropy}\index{Entropy} $\sigma(N,U)$ is given by 
	\begin{align}
		\sigma(N,U) = \log g(N,U).
	\end{align}
	The fundamental temperature $\tau$ is defined by the relation
	\begin{align}
		\frac{1}{\tau} = \left(\frac{\partial \sigma}{\partial U}\right)_{N,V}.
	\end{align}
	\section{Boltzmann Distribution and Helmholtz Free Energy}
	The \textbf{partition function}\index{Partition function} Z is
	\begin{align}
		Z \equiv \sum_{s} \exp\left(-\frac{\epsilon_s}{\tau}\right).
	\end{align}
	The probability of finding a system of N particles in a state s of energy $\epsilon_s$ when the system is in thermal contact with a large reservoir at temperature $\tau$ is
	\begin{align}
		P(\epsilon_s) = \frac{1}{Z}\exp\left(-\frac{\epsilon_s}{\tau}\right).
	\end{align} 
	The pressure is given by
	\begin{align}
		P = -\left(\frac{\partial U}{\partial V}\right)_\sigma = \tau\left(\frac{\partial \sigma}{\partial V}\right)_U.
	\end{align}
	The \textbf{Helmholtz Free Energy}\index{Helmholtz Free Energy} is a minimum in equilibrium for a system held at constant $\tau, V$ and is defined as
	\begin{align}
		F&\equiv U-\tau \sigma = -\tau\log(Z) \\
		dF&=-\sigma d\tau-pdV+\mu dN.	
	\end{align}
	From this we have
	\begin{align}
		\sigma &= -\left(\frac{\partial F}{\partial \tau}\right)_V \\
		P &= -\left(\frac{\partial F}{\partial V}\right)_\tau \\
		\mu &= \left(\frac{\partial F}{\partial N}\right)_{\tau,V}.
	\end{align}
	For an ideal monotonic gas of N atoms of spin zero with $n=N/V << n_Q$,
	\begin{align}
		Z_n = \frac{Z_1^N}{N!}=\frac{(n_Q V)^N}{N!},
	\end{align}
	The quantum concentration $n_Q$ is defined by
	\begin{align}
		n_Q \equiv \left(\frac{M\tau}{2\pi\hbar^2}\right)^{3/2}.
	\end{align}
	Furthermore, for an ideal gas we have
	\begin{align}
		PV&=N\tau \implies P = n\tau\\
		\sigma &=N\left[\log\left(\frac{n_Q}{n}\right)+\frac{5}{2}\right] \\
		C_V&=\frac{3}{2}N \hspace{0.5cm}\textrm{and}\hspace{0.5cm}C_p=\frac{5}{2}N.  
	\end{align}
	The thermal average energy of an atom a a box is
	\begin{align}
		U = \langle \epsilon \rangle = \tau^2 \frac{\partial \log (Z_1)}{\partial \tau}.
	\end{align}
	For a system of fixed volume in thermal contact with a reservoir, the mean square fluctuation in energy of the system is
	\begin{align}
		\langle(\epsilon - 	\langle\epsilon \rangle)^2 \rangle = \tau^2\left(\frac{\partial U}{\partial \tau}\right)_V
	\end{align}  
	\section{Thermal Radiation and Planck Distribution}
	The \textbf{Planck distribution function}\index{Planck!distribution function} for the thermal average number of photons in a cavity mode of frequency $\omega$ is
	\begin{align}
		\langle s \rangle = \frac{1}{\exp(\hbar \omega/\tau)-1}.
	\end{align}
	The \textbf{Stefan-Boltzmann law}\index{Stefan-Boltzmann law} for the radiant energy density in a cavity at temperature $\tau$ is
	\begin{align}
		\frac{U}{V} = \frac{\pi^2\tau^4}{15\hbar^3 c^3} \implies \frac{\sigma}{V} = \frac{4\pi^2\tau^3}{45\hbar^3 c^3}.	
	\end{align}
	The \textbf{Planck radiation law}\index{Planck!radiation law} for the energy per unit volume per unit range of frequency is
	\begin{align}
		u_\omega = \frac{\hbar}{\pi^2c^3}\frac{\omega^3}{\exp(\hbar \omega/\tau)-1}.
	\end{align}
	The flux density of radiant energy $J_\nu$ and the Stefan-Boltzmann constant are
	\begin{align}
		J_\nu &= \sigma_BT^4 \\
		\sigma_B &= \frac{\pi^2k_B^4}{60\hbar^3c^3}.
	\end{align}
	The Debye\index{Debye} low temperature limit of the heat capacity of a dielectric solid (where $\vartheta$ is the Debye temperature) is, in conventional units,
	\begin{align}
		C_V &= \frac{12\pi^4Nk_B}{5}\left(\frac{T}{\vartheta}\right)^3 \\
		\vartheta &= \frac{\hbar c}{k_B}\left(\frac{6\pi^2N}{V}\right)^{1/3}.
	\end{align}
	The \textbf{Debye concentration} is
	\begin{align}
		n_D=n_{max} = \left(\frac{6N}{\pi}\right)^{1/3}
	\end{align}
	\section{Chemical Potential and Gibbs Distribution}
	The \textbf{chemical potential}\index{Chemical potential} is defined as follows. Two systems are in diffusive equilibrium if $\mu_1=\mu_2$.
	\begin{align}
		\mu(\tau,V,N)&\equiv \left(\frac{\partial F}{\partial N}\right)_{\tau,V} \\ \mu 	&= \left(\frac{\partial U}{\partial N}\right)_{\sigma,V} = -\tau\left(\frac{\partial \sigma}{\partial N}\right)_{U,V}.
	\end{align}
	The chemical potential is made up of two parts, external and internal. The external part is the potential energy of a particle in an external field of force. The internal part is of thermal origin; for an ideal \textbf{monatomic gas}\index{Monatomic gas}
	\begin{align}
		\mu(int) = \tau \log\left(\frac{n}{n_Q}\right).
	\end{align}
	The \textbf{Gibbs factor}\index{Gibbs!Factor} gives the probability that a system at chemical potential $\mu$ and temperature $\tau$ will have N particles and be in a quantum state s of energy $\epsilon_s$.
	\begin{align}
		P(N,\epsilon_s) = \frac{1}{\mathscr{Z}}\exp\left[\frac{N\mu-\epsilon_{s}}{\tau}\right]. 
	\end{align} 
	The \textbf{Gibbs sum}\index{Gibbs!Sum} is taken over all states for all numbers of particles.
	\begin{align}
	\mathscr{Z} \equiv \sum_{ASN} \exp\left[\frac{N\mu-\epsilon_{s(N)}}{\tau}\right].
	\end{align}
	The \textbf{absolute activity}\index{Absolute activity} $\lambda$ is defined by
	\begin{align}
	\lambda \equiv \exp\left[\frac{\mu}{\tau}\right].
	\end{align}
	The thermal average number of particles is
	\begin{align}
	\langle N \rangle = \lambda \frac{\partial}{\partial \lambda}\log(\mathscr{Z}).
	\end{align}
	For a system in diffusive contact with a reservoir, the number of particles is not constant and
	\begin{align}
	\langle N \rangle &= \frac{\tau}{\mathscr{Z}}\left(\frac{\partial \mathscr{Z}}{\partial \mu}\right)_{\tau,V} \\
	\langle N^2 \rangle &= \frac{\tau^2}{\mathscr{Z}}\left(\frac{\partial^2 \mathscr{Z}}{\partial \mu^2}\right).
	\end{align}
	\section{Ideal Gas}
	The Fermi-Dirac (+) and Bose-Einstein (-) distribution functions are
	\begin{align}
	f(\epsilon) = \frac{1}{\exp[(\epsilon-\mu)/\tau]\pm 1}
	\end{align}
	Occupancy of an orbital in the classical limit of $f(\epsilon) << 1$
	\begin{align}
	f(\epsilon) = \lambda \exp(-\epsilon/\tau).
	\end{align}
	Given N, we can determine $\lambda$ in the classical limit as
	\begin{align}
	\lambda = \frac{N}{\sum \exp(-\epsilon_n/\tau)} = \frac{N}{n_Q V}.
	\end{align}
	The energy of a free particle orbital of quantum number n in a cube of volume V is
	\begin{align}
	\epsilon_n = \frac{1}{2M}\left(\frac{n \pi \hbar}{V^{1/3}}\right)^2.
	\end{align}
	A useful transformation from the summation to the integral follows as
	\begin{align}
	\sum_n e^{-\epsilon_n/\tau} = \frac{\pi}{2}\int n^2 e^{-\epsilon_n/\tau} dn.
	\end{align}
	Some important relationships are
	\begin{align}
	F &= \int \mu dN = N\tau[\log(n/n_Q)-1] \\
	P&= -\left(\frac{\partial F}{\partial V}\right)_{\tau,N} = \frac{N\tau}{V}.
	\end{align}
	For an ideal gas, the \textbf{entropy}\index{Entropy} (with $\sigma_1$ beign a constant independent of $\tau$ and V)is
	\begin{align}
	\sigma = C_v \log(\tau)+N\log(V)+\sigma_1.
	\end{align}
	The average pressure in a system in thermal contact with a heat reservoir is given by 
	\begin{align}
	P= -\frac{1}{Z}\sum_s \left(\frac{\partial \epsilon_s}{\partial  V}\right)_N e^{-\epsilon_s/\tau}.
	\end{align}
	For a gas of free particles
	\begin{align}
	\left(\frac{\partial \epsilon_s}{\partial  V}\right)_N = -\frac{2\epsilon_s}{3V}.
	\end{align}
	The Gibbs sum for an ideal gas with identical atoms and the probability that there are N atoms in the gas in a volume V in diffusive contact with a reservoir is
	\begin{align}
	\mathcal{Z} &= \exp(\lambda n_Q V) \\
	P(N) &= \frac{\langle N \rangle^N}{N!} e^{-\langle N \rangle}.
	\end{align}
	
	\section{Heat and Work}
	\textbf{Heat}\index{Heat} is the transfer of energy by thermal contact with a reservoir. In a reversible process
	\begin{align}
		dQ = \tau d\sigma.
	\end{align} 
	The \textbf{Carnot Energy}\index{Carnot!Energy} conversion efficiency $\eta_C$ is the upper limit to the ratio $W/Q_h$ of the work generated to the heat added:
	\begin{align}
		\eta_C = \left(\frac{W}{Q_h}\right)_{rev} = \frac{(\tau_h-\tau_l)}{\tau_h}=\frac{(T_h-T_l)}{T_h}.
	\end{align}
	The \textbf{Carnot coefficient}\index{Carnot!Coefficient} of refrigerator performance is the upper limit of $Q_l/W$ of the heat extracted to the work consumed:
	\begin{align}
		\gamma_C = \left(\frac{Q_l}{W}\right)_{rev} = \frac{\tau_l}{\tau_h-\tau_l} = \frac{T_l}{T_h-T_l}.
	\end{align} 
	The effective work performed on a system at constant temperature and pressure in a reversible process is equal to the change in the Gibbs free energy:
	\begin{align}
		G\equiv U-\tau \sigma+pV. 
	\end{align}
	The chemical work performed on a system in the reversible transfer of dN particles to the system is 
	\begin{align}
		W_\mu = \mu dN.
	\end{align} 
	The change in the free energy density of a superconductor of type I caused by an external magnetic field is $B^2/	2\mu_0$ in SI units and $B^2 8 \pi$ in CGS units.
	\section{Gibbs Free Energy and Chemical Reactions}
	From the \textbf{Gibbs free energy}\index{Gibbs!Free energy}, we have
	\begin{align}
		dG=\mu dN-\sigma d\tau + Vdp.
	\end{align}
	From this we can determine the following relations:
	\begin{align}
		\mu &= \left(\frac{\partial G}{\partial N}\right)_{\tau, p}  \\
	-\sigma &= \left(\frac{\partial G}{\partial \tau}\right)_{N, p} \\
	V &= \left(\frac{\partial G}{\partial p}\right)_{N, \tau}.
	\end{align}
	The Gibbs Free energy is related to chemical potential via
	\begin{align}
		G(\tau, p, V) = N \mu(\tau,p).
	\end{align}
	The \textbf{Law of mass action}\index{Law of!mass action} for a chemical reaction is
	\begin{align}
		K(\tau) = \prod_{j}n_j^{\mu_j}
	\end{align}
	The \textbf{Maxwell Relations}\index{Maxwell!Relations} are
	\begin{align}
		\left(\frac{\partial V}{\partial \tau}\right)_{p,N} &=	-\left(\frac{\partial \sigma}{\partial p}\right)_{\tau,N} \\
		\left(\frac{\partial V}{\partial N}\right)_p &=	\left(\frac{\partial \mu}{\partial p}\right)_N \\
		\left(\frac{\partial \mu}{\partial \tau}\right)_N &=	-\left(\frac{\partial \sigma}{\partial N}\right)_\tau \\
	\end{align}
	The \textbf{acidity}\index{Acidity} or alkalinity of a solution in terms of pH is defined as
	\begin{align}
		pH \equiv -\log_{10}[H^+].
	\end{align}
	The \textbf{reaction quotient}\index{Reaction quotient} for the chemical reaction $\alpha A+\beta B \rightleftharpoons \gamma C + \delta D$ is
	\begin{align}
		Q = \frac{[C]^\gamma [D]^\delta}{[A]^\alpha[B]^\beta}.
	\end{align}
	\section{Phase Transformations}
	\textbf{Enthalpy}\index{Enthalpy} is defined as
	\begin{align}
		H \equiv U - pV.
	\end{align}
	The coexistence curve in the $p-\tau$ plane between two phases must satisfy the \textbf{Clausius-Clapeyron equation}\index{Clausius-Clapeyron equation} (L is the latent heat $L=H_1-H_2$)
	\begin{align}
		\frac{dp}{d\tau} = \frac{L}{\tau \Delta v}.
	\end{align}
	The \textbf{van der Waals}\index{Van der Waals} equation of state is
	\begin{align}
		(p+N^2a/V^2)(V-Nb) = N\tau.
	\end{align}
	The critical points for a van der Waals gas are defined as
	\begin{align}
		\tau_c &= \frac{8a}{27b} \\
		p_c &= \frac{a}{27b^2} \\
		V_c &= 3Nb.
	\end{align}	
	\subsection{Kinetic Theory}
	The \textbf{Maxwell distribution}\index{Maxwelldistribution} describes the probability that an atom has a velocity $v$ in $dv$,
	\begin{align}
		P(v)dv = 4\pi\left(\frac{M}{2\pi\tau}\right)^{3/2}v^2e^{\frac{-Mv^2}{2\tau}}dv
	\end{align}
	The root mean square velocity is given by
	\begin{align}
		v_{rms} = \sqrt{\langle v^2 \rangle} = \sqrt{\frac{3\tau}{M}}.
	\end{align}
	The most probably value of the speed is
	\begin{align}
		v_{mp} = \sqrt{\frac{2\tau}{M}}.
	\end{align}
	The mean speed is
	\begin{align}
		\bar{c} = \langle v \rangle = \int_0^\infty v P(v) dv = \sqrt{\frac{8\tau}{\pi M}}
	\end{align}
	\textbf{Diffusion}\index{Diffusion} is described by the mean free path $\ell$ and the mean speed $\bar{c}\equiv \langle v \rangle$,
	\begin{align}
		\vec{J}_n=-D\nabla n, \hspace{1cm} D = \frac{1}{3}\bar{c}\ell
	\end{align}
	\textbf{Thermal conductivity}\index{Thermal conductivity} is described by the specific heat per unit volume $\bar{C}_V$,
	\begin{align}
		\vec{J}_\mu = -K \nabla \tau, \hspace{1cm} K = D\bar{C}_V =  \frac{1}{3}\bar{C}_V \bar{c}\ell
	\end{align}
	The coefficient of \textbf{viscosity}\index{viscosity} is given in terms of the mass density (where d is the molecular diameter) as
	\begin{align}
		\eta = D\rho = \frac{1}{3}\rho \bar{c} \ell = \frac{M\bar{c}}{3\pi d^2}
	\end{align}
	The electrical conductivity of a Fermi gas is given in terms of the relaxation time $\tau_c$,
	\begin{align}
		\sigma = \frac{nq^2\tau_c}{m}
	\end{align} 
	The electric current density is
	\begin{align}
		\vec{J}_q = \sigma\vec{E}
	\end{align}
	The \textbf{Wiedemann-Franz ratio}\index{Wiedemann-Franz ratio} holds for a classical gas of particles of charge q,
	\begin{align}
		\frac{K}{\tau \sigma}=\frac{3}{2q^2},\hspace{0.25cm} \textrm{ or }\hspace{0.25cm} \frac{K}{T\sigma} = \frac{3k_B^2}{2q^2}.
	\end{align}
	The \textbf{Boltzmann transport equation}\index{Boltzmann transport equation} in the relaxation time approximation is
	\begin{align}
		\frac{\partial f}{\partial t} + \alpha \nabla_V f+v\nabla_r f=-\frac{f-f_0}{\tau_c} 
	\end{align}
\end{multicols}
\begin{fancybox}[Table from chapter 14 of Kittel and Kroemer's ``Thermal Physics'' \cite{bib:Kittel_ThermalPhysics}.]{}
	\begin{center}
	\begin{tabular}{|c|c|c|c|c|c|c|}
		\hline
		Effect&\multicolumn{1}{|p{2cm}|}{\centering Flux of \\ Particle Property}&Gradient&Coefficient&Law&\multicolumn{1}{|p{2cm}|}{\centering Name of \\Law}&\multicolumn{1}{|p{2cm}|}{\centering Expression for \\ Coefficient}\\
		\hline
		Diffusion&\multicolumn{1}{|p{2cm}|}{\centering Particle \\Number}&$\nabla n$&Diffusivity, $D$&$\mathbf{J}_n = -D\nabla n$&\multicolumn{1}{|p{2cm}|}{\centering Fick's \\Law}&$D = \frac{\lambda <v>}{3}$\\
		\hline
		\multicolumn{1}{|p{2cm}|}{\centering Electrical \\Conductivity}&Charge&$\nabla \phi$&Conductivity, $\sigma$&$\mathbf{J}_q = -\sigma \nabla \phi$& \multicolumn{1}{|p{2cm}|}{\centering Ohm's \\Law}&$\sigma = \frac{n q^2 \lambda}{M <v>}$\\
		\hline
		\multicolumn{1}{|p{2cm}|}{\centering Thermal \\Conductivity}&Energy&$\nabla \tau$& \multicolumn{1}{|p{2cm}|}{\centering Thermal \\Conductivity, $\kappa$}&$\mathbf{J}_U = -\kappa \nabla \tau$&\multicolumn{1}{|p{2cm}|}{\centering Fourier's \\Law}& $\kappa = \frac{\hat{C_v} \lambda <v>}{3}$\\
		\hline
		Viscosity&\multicolumn{1}{|p{2cm}|}{\centering Transverse \\Momentum}&$\frac{d v_x}{dz}$&Viscosity, $\eta$&$\mathbf{J}_{p_x} = -\eta \frac{d v_x}{dz}$&	\multicolumn{1}{|p{2cm}|}{\centering Newtonian \\ Viscosity}&$\eta = \frac{M <v>}{3 \pi d^2}$\\
		\hline
	\end{tabular}
\end{center}
\end{fancybox}
