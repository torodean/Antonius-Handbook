\chapter{Electronics}
\thispagestyle{fancy}
\section{Electronic Symbols \& Circuit Diagrams}
Circuit diagrams are a major part of understanding and representing electronic circuits. Some common \textbf{circuit diagram symbols} follow:

\vspace{0.5cm}

\begin{circuitikz} 
	\draw 
	(0,0) to[ammeter, o-o, l=ammeter] (1.75,0)
	(1.75,0) to[voltmeter, o-o, l=voltmeter] (3.75,0)
	(3.75,0) to[ohmmeter, o-o, l=ohmmeter] (5.75,0)
	(5.75,0) to[lamp, o-o, l=lamp] (7,0)
	(7,0) to[american voltage source, o-o, l=voltage source] (9.5,0)
	(9.5,0) to[american current source, o-o, l=current source] (12.5,0)
	(12.5,0) to[sV, o-o, l=sinusoidal voltage source] (17,0)
	;
\end{circuitikz}

\begin{circuitikz}
	\draw 
	(0,0) to[sqV, o-o, l=square voltage source] (4,0)
	(4,0) to[esource, o-o, l=empty voltage source] (8,0)
	(8,0) to[dcisource, o-o, l=DC current source] (11.5,0)
	(11.5,0) to[dcvsource, o-o, l=DC voltage source] (15,0)
	(15,0) to[vsourcetri, o-o, l=$\Delta$ voltage source] (18,0)
	;
\end{circuitikz}

\begin{circuitikz}
	\draw
	(0,0) to[R, o-o, l=resistor] (2,0)
	(2,0) to[vR, o-o, l=variable resistor] (5,0)
	(5,0) to[pR, o-o, l=potentiometer] (8,0)
	(8,0) to[fuse, o-o, l=fuse] (10,0)
	(10,0) to[afuse, o-o, l=asymmetric fuse] (13,0)
	(13,0) to[Do, o-o, l=empty diode] (15.5,0)
	(15.5,0) to[D*, o-o, l=full diode] (18,0)	
	;
\end{circuitikz}

\begin{circuitikz}
	\draw
	(0,0) to[C, o-o, l=capacitor] (2,0)
	(2,0) to[vC, o-o, l=variable capacitor] (5,0)
	(5,0) to[L, o-o, l=inductor] (7,0)
	(7,0) to[vL, o-o, l=variable inductor] (10,0)
	(10,0) to[american inductor, o-o, l=inductor] (12,0)
	(12,0) to[variable american inductor, o-o, l=variable inductor] (15,0)
	(15,0) to[transmission line, o-o, l=transmission line] (18,0)
	;
\end{circuitikz}

\begin{circuitikz}
	\draw
	(0,0) to[battery1, o-o, l=battery] (2,0)
	(2,0) to[battery, o-o, l=battery] (4,0)
	(4,0) to[switch, o-o, l=switch] (6,0)
	(6,0) to[push button, o-o, l=push button] (9,0)
	(9,0) to[amp, o-o, l=amplifier] (11,0)
	(11,0) to[vamp, o-o, l=VGA] (13,0)
	(13,0) -- (13,0.3) -- (14.5,0.3) node[pground]{} -- (15,0.3) -- (17.3,0.3) node[ground]{}
	;
	\draw (14.5,0.3)node[above]{protective ground};
	\draw (17.3,0.3)node[above]{ground};
\end{circuitikz}

An elementary building block of a circuit is a \textbf{logic gate}. At any given time, the 3 nodes of a logic gate are either true (1) or false (0). The common notation for the logic gates follow:

\vspace{0.5cm}

\begin{circuitikz}
	\draw (0,0) node[american and port]{};
	\draw (-0.7,0.7)node[above]{AND};
	\draw (2,0) node[american or port]{};
	\draw (1.3,0.7)node[above]{OR};
	\draw (4,0) node[american nand port]{};
	\draw (3.3,0.7)node[above]{NAND};
	\draw (6,0) node[american nor port]{};
	\draw (5.3,0.7)node[above]{NOR};
	\draw (7.5,0) node[american not port]{};
	\draw (7.3,0.7)node[above]{NOT};
	\draw (10,0) node[american xor port]{};
	\draw (9.3,0.7)node[above]{XOR};
	\draw (12,0) node[american xnor port]{};
	\draw (11.3,0.7)node[above]{XNOR};
\end{circuitikz}

\section{Equivalent Circuits}
When dealing with circuit diagrams, it is often helpful to simplify a circuit using an equivalent circuit. Some basic \textbf{circuit equivalences} follow:

\begin{circuitikz}
	\draw (0,0) to[battery1, i=I,  l=V] (0,2)
	(0,2) to[R, l=$R_1$] (2,2)
	(2,2) to[R, l=$R_2$] (2,0)
	(2,0) -- (0,0); 
	
	\draw (3.5,1) node[]{$\Longleftrightarrow$} (3.5,1);
	
	\draw (5,0) to[battery1, i=I, l=V] (5,2)
	(7,2) -- (7,0)
	(5,2) to[R, l={$R_1+R_2$}] (7,2)
	(7,0) -- (5,0); 
	
	\draw (9,0) to[battery1, i=I,  l=V] (9,2)
	(9,2) -- (10.5,2)
	(10.5,2) to[C, l=$C_1$] (10.5,0)
	(10.5,2) to[C, l=$C_2$] (12,2)
	(12,2) -- (12,0)
	(9,0) -- (12,0); 
	
	\draw (13,1) node[]{$\Longleftrightarrow$} (13,1);
	
	\draw (14.5,0) to[battery1, i=I, l=V] (14.5,2)
	(14.5,2) to[C, l=$C_1+C_2$] (16,2)
	(16,2) -- (16,0)
	(16,0) -- (14.5,0); 
\end{circuitikz}

\vspace{0.5cm}

\begin{circuitikz}
	\draw (0,0) to[battery1, i=I,  l=V] (0,2)
	(1.5,0) to[R, l=$R_1$] (1.5,2)
	(3,0) to[R, l=$R_2$] (3,2)
	(0,0) -- (3,0)
	(0,2) -- (3,2); 
	
	\draw (4,1) node[]{$\Longleftrightarrow$} (4,1);
	
	\draw (5.5,0) to[battery1, i=I, l=V] (5.5,2)
	(5.5,2) to[R, l=\hspace{0.5cm}$\left(\frac{1}{R_1}+\frac{1}{R_2}\right)^{-1}$] (7.5,2)
	(7.5,2) -- (7.5,0)
	(7.5,0) -- (5.5,0); 
	
	\draw (10,0) to[battery1, i=I,  l=V] (10,2)
	(10,2) to[C, l=$C_1$] (11.5,2)
	(11.5,2) to[C, l=$C_2$] (11.5,0)
	(11.5,0) -- (10,0); 
	
	\draw (13,1) node[]{$\Longleftrightarrow$} (13,1);
	
	\draw (14.5,0) to[battery1, i=I, l=V] (14.5,2)
	(14.5,2) to[C, l=\hspace{0.5cm}$\left(\frac{1}{C_1}+\frac{1}{C_2}\right)^{-1}$] (16,2)
	(16,2) -- (16,0)
	(16,0) -- (14.5,0); 
\end{circuitikz}

\vspace{0.5cm}
\begin{multicols}{2}
A \textbf{voltage divider}\index{Voltage divider}:
\begin{center}
\begin{circuitikz}
	\draw (0,0) to[battery1, i=I,  l=V] (0,2)
	(0,2) to[R, l=$R_1$] (2,2)
	(2,2) to[R, l=$R_2$] (2,0)
	(2,0) -- (0,0)
	(2,2) -- (3.5,2)
	(2,0) -- (3.5,0); 
	\draw[-latex] (3.5,1.3) -- (3.5,1.9) {};
	\draw[-latex] (3.5,0.7) -- (3.5,0.1) {};
	\draw (3.5,1) node[]{$V_2$} (3.5,1);	
\end{circuitikz}
\end{center}

\begin{align}
V_2 &= \frac{VR_2}{R_1+R_2} \\
V &= I_1(R_1+R_2) = I_2(R_1+R_2)
\end{align}

A \textbf{current divider}\index{Current divider}:
\begin{center}
	\begin{circuitikz}
		\draw (0,0) to[american current source,  l=I] (0,2)
		(1.5,0) to[R, l=$R_1$] (1.5,2)
		(3,0) to[R, l=$R_2$] (3,2)
		(0,0) -- (3,0)
		(0,2) -- (3,2)
		(2,0) -- (0,0);
		\draw[-latex] (3.4,1.5) -> (3.4,0.5);
		\draw (3.7,1) node[]{$I_2$} (3.5,1);	
	\end{circuitikz}
\end{center}

\begin{align}
I_2 &= \frac{I R_1  }{R_1+R_2}
\end{align}
\end{multicols}

Electrical \textbf{impedance}\index{Impedance} is "the total opposition to alternating current by an electric circuit." \cite{bib:dictionary}
\begin{align}
	Z_{\textrm{series}} = \sum_{i} Z_{i} \andspace{1cm} Z_{\textrm{parallel}} = \left[\sum_{i} \frac{1}{Z_i}\right]^{-1}.
\end{align}
The electrical impedance for a resistor, capacitor, and inductor is given by
\begin{align}
Z_{\textrm{resistor}} = R \andspace{1cm} Z_{\textrm{capacitor}} = \frac{1}{i \omega C} \andspace{1cm} Z_{\textrm{inductor}} = i \omega L. 
\end{align}
The complex form of \textbf{Ohm's Law}\index{Ohms law} can be written in terms of \textbf{phasers}\index{Phaser} (denoted with a tilde overhead, i.e \~{X}): $\textrm{\~{V}}\equiv \textrm{\~{I}}Z$.
For cases with an alternating power source, we can use the phasers to find the voltage and current at a given time
\begin{align}
V(t) \equiv \mathfrak{R}[\textrm{\~{V}}e^{i \omega t}] \andspace{1cm} I(t) \equiv \mathfrak{R}[\textrm{\~{I}}e^{i \omega t}]
\end{align}

\begin{multicols}{2}
	A \textbf{low pass filter}\index{Low pass filter}:
	\begin{center}
		\begin{circuitikz}
			\draw (0,0) to[battery1, i=I,  l=$V_{in}$] (0,2)
			(0,2) to[R, l=$R$] (2,2)
			(2,2) to[C, l=$C$] (2,0)
			(2,0) -- (0,0)
			(2,2) -- (3.5,2)
			(2,0) -- (3.5,0); 
			\draw[-latex] (3.5,1.3) -- (3.5,1.9) {};
			\draw[-latex] (3.5,0.7) -- (3.5,0.1) {};
			\draw (3.5,1) node[]{$V_{out}$} (3.5,1);	
		\end{circuitikz}
	\end{center}
	A \textbf{high pass filter}\index{High pass filter}:
	\begin{center}
		\begin{circuitikz}
			\draw (0,0) to[battery1, i=I,  l=$V_{in}$] (0,2)
			(0,2) to[C, l=$C$] (2,2)
			(2,2) to[R, l=$R$] (2,0)
			(2,0) -- (0,0)
			(2,2) -- (3.5,2)
			(2,0) -- (3.5,0); 
			\draw[-latex] (3.5,1.3) -- (3.5,1.9) {};
			\draw[-latex] (3.5,0.7) -- (3.5,0.1) {};
			\draw (3.5,1) node[]{$V_{out}$} (3.5,1);	
		\end{circuitikz}
	\end{center}
\end{multicols}