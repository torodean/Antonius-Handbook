\chapter{General Mathematics}
\thispagestyle{fancy}
Definitions
\begin{multicols}{2}
	\noindent
	\begin{align}
	\sin(x) &=\frac{1}{2i}(e^{ix}-e^{-ix}) \\
	\sinh(x) &= \frac{1}{2}(e^{x}-e^{-x}) \\
	&= -i\sin(ix)
	\end{align}
	\begin{align}
	\cos(x) &=\frac{1}{2}(e^{ix}+e^{-ix}) \\
	\cosh(x) &=\frac{1}{2}(e^{x}+e^{-x}) \\
	&= \cos(ix)
	\end{align}
\end{multicols}
\textbf{Curl Theorem:}\index{Curl Theorem} A special case of Stokes' theorem\index{Stokes' theorem} in which $\vec{F}$ is a vector field and $M$ is an oriented, compact embedded 2-manifold with boundary in $\mathbb{R}^3$, and a generalization of Green's theorem from the plane into three-dimensional space. The curl theorem states 
\begin{align}
	\int_S (\nabla \times \vec{F})\cdot d\vec{a} = \int_{\partial S}\vec{F} \cdot d\vec{s}
\end{align}
\textbf{Green's theorem}\index{Green's theorem} is a vector identity which is equivalent to the curl theorem 
\begin{align}
	\iint_S \left(\frac{\partial Q}{\partial x}-\frac{\partial P}{\partial y}\right) dxdy = \oint_{\partial S} P(x,y)dx+Q(x,y)dy
\end{align}
The \textbf{divergence theorem}\index{Divergence theorem} is also known as Gauss's theorem (e.g., Arfken 1985) or the Gauss-Ostrogradsky theorem. Let $V$	 be a region in space with boundary $\partial V$. Then the volume integral of the divergence $\nabla \cdot \vec{F}$ of $\vec{F}$ over $V$ and the surface integral of $\vec{F}$ over the boundary $\partial V$ of $V$ are related by 
\begin{align}
	\int_V (\nabla \cdot \vec{F}) dV = \int_{\partial V}\vec{F} \cdot d\vec{a}
\end{align}
The \textbf{gradient theorem}\index{Gradient theorem} (where the integral is a line integral) is
\begin{align}
	\int_a^b (\nabla f) \cdot d\vec{s} = f(b)-f(a)
\end{align}
The \textbf{Gamma function}\index{Gamma function} $\Gamma$ and the \textbf{Riemann zeta function} $\zeta$ are given by
\begin{align}
\Gamma(z) &\equiv\int_{0}^{\infty}t^{z-1}e^{-t}dt \\
\zeta(z)&=\sum_{k=1}^{\infty}\frac{1}{k^z}\implies \zeta'(z)=-\sum_{k=1}^{\infty}\frac{\ln(k)}{k^z} \\
\zeta(z)\Gamma(z)&=\int_{0}^{\infty}\frac{u^{z-1}}{e^u-1}du
\end{align}
The most general case of the \textbf{binomial theorem} \index{Binomial theorem} is the binomial series identity 
\begin{align}
(x+y)^n &= \sum_{i=1}^{n} {{n}\choose{k}}x^{n-k}y^{k}
\end{align}
The \textbf{binomial coefficient} is defined as follows, with Pascals Formula implied.
\begin{align}
_nC_r\equiv{{n}\choose{k}} &\equiv \frac{n!}{(n-k)!k!} \equiv \frac{\Gamma(n+1)}{\Gamma(k+1)\Gamma(n-k+1)} \\
{{n}\choose{k}}&={{n-1}\choose{k-1}}+{{n-1}\choose{k}}
\end{align}
The general formula for the power sum of the first $n$ positive integers, 
\begin{align}
\sum_{k=1}^{n}k^p=\frac{1}{p+1}\sum_{i=1}^{p+1}(-1)^{\delta_{ip}}{{p+1}\choose{i}}B_{p+1-i}n^i,
\end{align}
where $\delta_{ip}$ is the Kronecker delta  and $B_i$ is the $i$th Bernoulli number. The Bernoulli numbers $B_n$ are a sequence of signed rational numbers that can be defined by the exponential generating function 
\begin{align}
\frac{x}{e^x-1}\equiv\sum_{n=0}^{\infty}\frac{B_nx^n}{n!}.
\end{align}
The simplest interpretation of the \textbf{Kronecker delta}\index{Kronecker delta} is as the discrete version of the delta function defined by 
\begin{align}
\delta_{ij}\equiv
\begin{cases}
0 & \textrm{for } i \neq j \\
1 & \textrm{for } i = j.
\end{cases}
\end{align}
Determining the sums for the first few terms of the power sum give
\begin{align}
\sum_{i=0}^{n}i   &=\frac{1}{2}n(n+1) \\
\sum_{i=0}^{n}i^2 &=\frac{1}{6}(2n^3+3n^2+n)=\frac{1}{6}n(2n+1)(n+1) \\
\sum_{i=0}^{n}i^3 &=\frac{1}{4}(n^4+2n^3+n^2)=\frac{1}{4}n^2(n+1)^2 \\
\sum_{i=0}^{n}i^4 &=\frac{1}{30}(6n^5+15n^4+10n^3-n) \\
\sum_{i=0}^{n}i^5 &=\frac{1}{12}(2n^6+6n^5+5n^4-n^2).
\end{align}
A \textbf{Taylor series}\index{Taylor series} is an expansion of a function about a point. A one-dimensional Taylor series of a real function $f(x)$ about the point x=a is given by 
\begin{align}
f(x)&=f(a)+f'(a)(x-a)+\frac{1}{2!}f''(a)(x-a)^2+\frac{1}{3!}f'''(a)(x-a)^4+\cdots 
\end{align}
Some common series expansions include: 
\begin{align}
e^x&=\sum_{n=0}^{\infty} \frac{x^n}{n!}=1+x+\frac{1}{2!}x^2 +\frac{1}{3!}x^3 +\cdots \\ 
\ln(1+x)&=\sum_{n=0}^{\infty} \frac{(-1)^{n}x^{n+1}}{n+1}=x-\frac{1}{2}x^2 +\frac{1}{3}x^3 -\cdots\hspace{0.5cm} \textrm{with }|x|<1\\ 
\sin(x)&=\sum_{n=0}^{\infty}\frac{(-1)^{n}x^{2n+1}}{(2n+1)!} = x-\frac{x^3}{3!}+\frac{x^5}{5!}-\cdots \\
\cos(x)&=\sum_{n=0}^{\infty}\frac{(-1)^{n}x^{2n}}{(2n)!} = 1-\frac{x^2}{2!}+\frac{x^4}{4!}-\cdots \\
\tan(x)&= x+\frac{1}{3}x^3+\frac{2}{15}x^5+\cdots[|x|<\pi/2] \\
\sinh(x)&=\sum_{n=0}^{\infty}\frac{x^{2n+1}}{(2n+1)!} = x+\frac{x^3}{3!}+\frac{x^5}{5!}+\cdots \\
\cosh(x)&=\sum_{n=0}^{\infty}\frac{x^{2n}}{(2n)!} = 1+\frac{x^2}{2!}+\frac{x^4}{4!}+\cdots \\
\tanh(x)&= x-\frac{1}{3}x^3+\frac{2}{15}x^5-\cdots[|x|<\pi/2]\\
(1+x)^n&=1+nx+\frac{n(n-1)}{2!}x^2+\cdots[|x| < 1]
\end{align}	
\textbf{Dirac Delta Function:}\index{Dirac Delta Function} The delta function is a generalized function that can be defined as the limit of a class of delta sequences.
\begin{align}
	\delta(x) = \frac{1}{\pi}\lim_{\epsilon\rightarrow 0}\frac{\epsilon}{x^2+\epsilon^2} = \frac{1}{2}\lim_{\epsilon\rightarrow 0}\epsilon|x|^{\epsilon-1} =\lim_{\epsilon\rightarrow 0} \frac{1}{\pi x}\sin\left(\frac{x}{\epsilon}\right) = \lim_{\epsilon\rightarrow 0^+}\frac{1}{2\sqrt{\pi \epsilon}}e^{-x^2/(4\epsilon)}
\end{align}
The Dirac delta can be thought of as a function on the real line which is zero everywhere except where the arguments of the function are zero, where it is infinite,
\begin{align}
	\delta(x) = 
	\begin{cases} 
		\infty & x=0 \\
			 0 & x \neq 0 
	\end{cases}
\end{align}
For any $\epsilon > 0$, the delta function has the fundamental property that 
\begin{align}
	\int_{-\infty}^{\infty}f(x)\delta(x-a)dx = f(a) \hspace{1cm}\textrm{and}\hspace{1cm}\int_{x-\epsilon}^{x+\epsilon}f(x)\delta(x-a)dx = f(a)
\end{align}
The fundamental equation that defines derivatives of the delta function $\delta(x)$ is
\begin{align}
	\int f(x)\delta^{(n)}(x)dx \equiv -\int \frac{\partial f}{\partial x}\delta^{(n-1)}(x)dx
\end{align} 
This implies 
\begin{align}
	x^n\delta^{(n)}(x)=(-1)^nn!\delta(x)
\end{align}
A few identities and common expressions using the delta function are
\begin{align}
	\int_{-\infty}^{\infty}f(x)\delta(ax)dx &= \frac{1}{|a|}f(0) \\
	\int_{-1}^{1}\delta\bigg(\frac{1}{x}\bigg)dx = 0
\end{align}




\section{Coordinate Systems}
\begin{multicols}{2}
Cylindrical coordinates\index{Cylindrical coordinates}
\begin{align}
r&=\sqrt{x^2+y^2} \\
x&=r\cos(\theta) \\
y&=r\sin(\theta) \\
z&=z \\
dV&= r \textrm{ dr d$\theta$ dz}
\end{align}
Spherical coordinates\index{Spherical coordinates}
\begin{align}
r&=\sqrt{x^2+y^2+z^2} \\
x&=r\sin(\theta)\cos(\phi) \\
y&=r\sin(\theta)\sin(\phi) \\
z&=r\cos(\theta) \\
dV&=r^2\sin(\theta)\textrm{dr d$\theta$ d$\phi$}
\end{align}
Polar Coordinates\index{Polar Coordinates}
\begin{align}
	r&=\sqrt{x^2+y^2} \\
	x&=r\cos(\theta) \\
	y&=r\sin(\theta) \\
	dA&= r \textrm{ dr d$\theta$}
\end{align}
Elliptic cylindrical coordinates\index{Elliptic cylindrical coordinates}
\begin{align}
	x&= a \cosh(u) \cos(v) \\
	y&= a \sinh(u) sin(v) \\
	z&= z \\
	dV&=a^2[\sinh^2(u)+\cosh^2(v)]\textrm{du dv dz}
\end{align}
\end{multicols}




\section{Vector Operations}
For any vector $\vec{r}=(r_1,r_2,\dots,r_n)$ in $n$-dimensions, the magnitude and unit vector is
\begin{align}
|\vec{r}|\equiv\sqrt{\vec{r}\cdot\vec{r}}=\sqrt{r_1^2+r_2^2+\cdots+r_n^2} \hspace{2cm} \hat{r}\equiv\frac{\vec{r}}{|\vec{r}|}
\end{align}
Dot and cross products for 3-dimensional vectors, where $\theta$ is the smallest angle between them, $\vec{r}=(r_x,r_y,r_z)$ and $\vec{s}=(s_x,s_y,s_z)$
\begin{align}
\vec{r}\cdot \vec{s} &=|\vec{r}||\vec{s}|\cos(\theta)=r_xs_x+r_ys_y+r_zs_z \\
\vec{r} \times \vec{s} &= |\vec{r}||\vec{s}|\sin(\theta)= (r_ys_z-r_zs_y, r_zs_x-r_xs_z, r_xs_y-r_ys_x) =  \begin{vmatrix}
\boldsymbol{\hat{x}} & \boldsymbol{\hat{y}} & \boldsymbol{\hat{z}} \\ 
r_x & r_y & r_z \\ 
s_x & s_y & s_z 
\end{vmatrix}
\end{align}
Triple product vector identities
\begin{align}
\vec{A}\cdot (\vec{B} \times \vec{C}) = \vec{B}\cdot (\vec{C}\times \vec{A})= \vec{C}\cdot(\vec{A}\times \vec{B}) &= -\vec{B}\cdot (\vec{A} \times \vec{C}) = -\vec{C}\cdot (\vec{B} \times \vec{A})=-\vec{A}\cdot (\vec{C} \times \vec{B}) \\
\vec{A}\times (\vec{B}\times \vec{C}) &=\vec{B}(\vec{A}\cdot\vec{C})-\vec{C}(\vec{A}\cdot\vec{B})
\end{align}
Product rule vector identities
\begin{align}
	\nabla (fg) &= f(\nabla g) + g(\nabla f) \\
	\nabla (\vec{A} \cdot \vec{B}) &= \vec{A} \times(\nabla \times \vec{B})+ \vec{B} \times (\nabla \times \vec{A}) + (\vec{A} \cdot \nabla)\vec{B} + (\vec{B} \cdot \nabla)\vec{A} \\
	\nabla \cdot (f\vec{A}) &= f(\nabla  \cdot \vec{A}) + \vec{A} \cdot (\nabla f) \\
	\nabla \cdot (\vec{A} \times \vec{B}) &= \vec{B} \cdot (\nabla \times \vec{A}) - \vec{A} \cdot (\nabla \times \vec{B}) \\
	\nabla \times (f\vec{A}) &= f(\nabla \times \vec{A})- \vec{A} \times (\nabla f) \\
	\nabla \times (\vec{A}\times\vec{B}) &= (\vec{B}\cdot \nabla)\vec{A}-(\vec{A}\cdot\nabla)\vec{B}+\vec{A}(\nabla \cdot \vec{B})-\vec{B}(\nabla\cdot\vec{A})
\end{align}
Second derivative vector identities
\begin{align}
	\nabla \cdot (\nabla \times \vec{A}) =0 \hspace{2cm} \nabla \times (\nabla f)=0 \hspace{2cm}
	\nabla \times (\nabla \times \vec{A}) = \nabla (\nabla \cdot \vec{A})- \nabla^2\vec{A}
\end{align}
A few other useful identities include
\begin{align}
	\nabla \cdot \frac{\hat{r}}{r^2} = 4\pi\delta^2(\vec{r}) \hspace{2cm}\nabla \times \frac{\hat{r}}{r^2} =\frac{\vec{r}}{r^3}= 0 \hspace{2cm} \nabla \cdot \frac{\hat{r}}{r} = \frac{1}{r^2}
\end{align}




\section{Triangles}
Let a triangle have side lengths $a$, $b$, and $c$ with opposite angles $A$, $B$, and $C$. 
\begin{multicols}{2}
The area of a triangle can be given by 
\begin{align}
A&=\sqrt{s(s-a)(s-b)(s-c)} \\
s&=(a+b+c)/2
\end{align}		
Law of Cosines:\index{Law of Cosines}
\begin{align}
c^2=a^2+b^2-2ab\cos(C)
\end{align}
Law of Sines:\index{Law of Sines}
\begin{align}
\frac{\sin(A)}{a}=\frac{\sin(B)}{b}=\frac{\sin(C)}{c}
\end{align}
Law of tangents:\index{Law of tangents}
\begin{align}
\frac{a-b}{a+b}=\frac{\tan((A-B)/2)}{\tan((A+B)/2)}
\end{align}
\end{multicols}
Mollweide's Formulas:\index{Mollweide's Formulas:}
\begin{align}
\frac{b-c}{a} &=\frac{\sin[(B-C)/2]}{\cos(A/2)} \\
\frac{c-a}{b} &=\frac{\sin[(C-A)/2]}{\cos(B/2)} \\
\frac{a-b}{c} &=\frac{\sin[(A_B)/2]}{\cos(C/2)}
\end{align}	










\newpage
\section{Trigonometric Identities}\index{Trigonometric Identities}
\begin{multicols}{2}
Pythagorean identities:
\begin{align}
1 &= \sin^2(\theta)+\cos^2(\theta)\\
1 &= \sec^2(\theta)-\tan^2(\theta) \\
1 &= \csc^2(\theta)-\cot^2(\theta) \\
1 &= \cosh^2(\theta)-\sinh^2(\theta) \\
1 &= \textrm{sech}^2(\theta)+\tanh^2(\theta)
\end{align}
Sum-Difference Formulas:
\begin{align}
\sin(\theta \pm \phi) &= \sin(\theta)\cos(\phi)\pm \cos(\theta)sin(\phi) \\
\cos(\theta \pm \phi) &= \cos(\theta)\cos(\phi)\mp \sin(\theta)sin(\phi) \\
\tan(\theta \pm \phi) &= \frac{\tan(\theta)\pm \tan (\phi)}{1 \mp \tan(\theta)\tan(\phi)}
\end{align}
Double Angle formulas:
\begin{align}
\sin(2\theta) &= 2\sin(\theta)\cos(\theta) \\
\cos(2\theta) &= \cos^2(\theta)-\sin^2(\theta)\\
&= 2\cos^2(\theta)-1 \\
&= 1 - 2\sin^2(\theta) \\
\tan(2\theta) &= \frac{2 \tan(\theta)}{1-\tan^2(\theta)}
\end{align}
Power-Reducing/Half Angle Formulas:
\begin{align}
\sin^2(\theta) &= \frac{1-\cos(2\theta)}{2}\\
\cos^2(\theta) &= \frac{1+\cos(2\theta)}{2}\\
\tan^2(\theta) &= \frac{1-\cos(2\theta)}{1+\cos(2\theta)}
\end{align}
Other relations
\begin{align}
\sin(-\theta) &=-\sin(\theta) \\
\cos(-\theta) &= \cos(\theta) \\
\sin(\theta\pm \pi/2) &= \pm \cos(\theta) \\
\sin(\theta\pm \pi) &= - \sin(\theta) \\
\cos(\theta\pm \pi/2) &= \mp \sin(\theta) \\
\cos(\theta\pm \pi) &= - \cos(\theta) 
\end{align}
Half-angle formulas
\begin{align}
\sin\bigg(\frac{\theta}{2}\bigg)=(-1)^{\theta/(2\pi)}\sqrt{\frac{1-\cos(\theta)}{2}} \\
\cos\bigg(\frac{\theta}{2}\bigg)=(-1)^{(\theta+\pi)/(2\pi)}\sqrt{\frac{1+\cos(\theta)}{2}}
\end{align}
The \textbf{Weierstrass substitution}\index{Weierstrass substitution} makes use of the half-angle formulas 
\begin{align}
\cos(\theta)=\frac{1-\tan^2(\theta/2)}{1+\tan^2(\theta/2)} \\
\sin(\theta)=\frac{2\tan(\theta/2)}{1+\tan^2(\theta/2)}
\end{align}
\end{multicols}
The half angle identity for tangent.
\begin{align}
\tan\bigg(\frac{\theta}{2}\bigg)=(-1)^{x/\pi}\sqrt{\frac{1-\cos(\theta)}{1+\cos(\theta)}} = \frac{\sin(\theta)}{1+\cos(\theta)}=\frac{1-\cos(\theta)}{\sin(\theta)}=\frac{\tan(\theta)\sin(\theta)}{\tan(\theta)+\sin(\theta)}
\end{align}
Other identities
\begin{align}
\cos(\theta)cos(\phi) &=\frac{1}{2}[\cos(\theta+\phi)+\cos(\theta-\phi)] \\
\sin(\theta)sin(\phi) &=\frac{1}{2}[\cos(\theta-\phi)-\cos(\theta+\phi)] \\
\sin(\theta)cos(\phi) &=\frac{1}{2}[\sin(\theta+\phi)+\sin(\theta-\phi)] \\
\cos(\theta)+\cos(\phi)&= 2\cos\bigg( \frac{\theta+\phi}{2}\bigg)\cos\bigg( \frac{\theta-\phi}{2}\bigg) \\
\cos(\theta)-\cos(\phi)&= 2\sin\bigg( \frac{\theta+\phi}{2}\bigg)\sin\bigg( \frac{\theta-\phi}{2}\bigg)
\end{align}
Multiple-angle formulas are given by 
\begin{align}
\sin(nx)&= \sum_{k=0}^{n}{{n}\choose{k}}\cos^k(x)\sin^{n-k}(x)\sin\big((n-k)\pi/2 \big) \\
\cos(nx)&= \sum_{k=0}^{n}{{n}\choose{k}}\cos^k(x)\sin^{n-k}(x)\cos\big((n-k)\pi/2 \big)
\end{align}








\section{Arbitrary Orthogonal Curvilinear Coordinates}\index{Orthogonal Coordinates}
A coordinate system composed of intersecting surfaces. If the intersections are all at right angles, then the curvilinear coordinates are said to form an orthogonal coordinate system. The scale factors are $h_i$,
\begin{align}
	\vec{a}_i &\equiv\frac{\partial \vec{r}}{\partial e_i} =  \frac{\partial x}{\partial e_i} \hat{x} + \frac{\partial y}{\partial e_i}\hat{y} + \frac{\partial x}{\partial e_i} \hat{z} = h_i \hat{e}_i = |\vec{a}_i| \hat{e}_i \\
	h_i &\equiv \left|\frac{\partial \vec{r}}{\partial e_i}\right|=|\vec{a}_i| = \sqrt{\frac{\partial x}{\partial e_i}+\frac{\partial y}{\partial e_i}+\frac{\partial z}{\partial e_i} } \\ \hat{e}_i &= \frac{1}{h_1}\frac{\partial \vec{r}}{\partial e_i}=\frac{\vec{a}_i}{|\vec{a}_i|}
\end{align}
The line element $d\vec{s}$ is determined by
\begin{align}
	d\vec{s}\equiv 
	d\vec{x}+d\vec{y}+d\vec{z} \equiv
	\vec{a}_1de_1+\vec{a}_2de_2+\vec{a}_3de_3
\end{align}
From this, $ds^2$ is given by
\begin{align}
	ds^2= d\vec{s}\cdot d\vec{s} = dx^2+dy^2+dz^2= h_1^2de_1^2+h_2^2de_2^2+h_3^2de_3^2
\end{align}
The differential vector and volume elements are therefore
\begin{align}
	d\vec{r} &= h_1 du_1 \hat{u}_1+h_2 du_2 \hat{u}_2+h_3 du_3 \hat{u}_3 \\
	dV &=h_1h_2h_3 du_1du_2du_3 = \left|\frac{\partial(x,y,z)}{\partial(u_1,u_2,u_3)}\right|du_1du_2du_3
\end{align}
The gradient in arbitrary curvilinear coordinates such that the gradient theorem is preserved:
\begin{align}
	\nabla f = \frac{1}{h_1}\frac{\partial f}{\partial x_1}\hat{x}_1+\frac{1}{h_2}\frac{\partial f}{\partial x_2}\hat{x}_2+\frac{1}{h_3}\frac{\partial f}{\partial x_3}\hat{x}_3
\end{align}
The divergence in arbitrary curvilinear coordinates such that the divergence theorem is preserved:
\begin{align}
	\nabla \cdot \vec{v} = \frac{1}{h_1h_2h_3}\left[\frac{\partial v_1}{\partial x_1}h_2h_3+\frac{\partial v_2}{\partial x_2}h_1h_3+\frac{\partial v_3}{\partial x_3}h_1h_2\right]
\end{align}
The Laplacian\index{Laplacian} for a scalar function $\phi$ (where the $h_i$ are the scale factors of the coordinate system - Weinberg 1972, p. 109; Arfken 1985, p. 92 \cite{bib:Wolfram}) is a scalar differential operator defined by
\begin{align}
	\nabla^2 \phi = \frac{1}{h_1h_2h_3} \bigg[	\frac{\partial}{\partial u_1} \bigg( \frac{h_2h_3}{h_1} \frac{\partial}{\partial u_1} \bigg) + \frac{\partial}{\partial u_2} \bigg( \frac{h_1h_3}{h_2} \frac{\partial}{\partial u_2} \bigg) + \frac{\partial}{\partial u_3} \bigg( \frac{h_1h_2}{h_3} \frac{\partial}{\partial u_3} \bigg) \bigg] \phi
\end{align}
The form of the Laplacian in several common coordinate systems (cartesian, cylindrical, parabolic, parabolic cylindrical, spherical and oblate spheroidal respectively) are
\begin{align}
	\nabla^2 f &= \frac{\partial^2 f}{\partial x^2}+ \frac{\partial^2 f}{\partial y^2}+ \frac{\partial^2 f}{\partial z^2} \\
	\nabla^2 f &=\frac{1}{r}\frac{\partial}{\partial r}\bigg(r \frac{\partial f}{\partial r}\bigg)+\frac{1}{r^2}\frac{\partial^2 f}{\partial \theta^2}+\frac{\partial^2 f}{\partial z^2} \\
	\nabla^2 f &= \frac{1}{uv(u^2+v^2)}\bigg[\frac{\partial}{\partial u}\bigg( uv\frac{\partial f}{\partial u} \bigg) + \frac{\partial}{\partial v}\bigg(u v \frac{\partial f}{\partial v}\bigg)\bigg]+\frac{1}{v^2u^2}\frac{\partial^2 f}{\partial \theta^2} \\
	\nabla^2 f &= \frac{1}{u^2+v^2}\bigg(\frac{\partial^2 f}{\partial u^2}+\frac{\partial^2 f}{\partial v^2} \bigg)+\frac{\partial^2 f}{\partial z^2} \\
	\nabla^2 f &= \frac{1}{r^2} \frac{\partial}{\partial r}\bigg(r^2 \frac{\partial f}{\partial r} \bigg)+\frac{1}{r^2 \sin^2 \phi}\frac{\partial^2 f}{\partial \theta^2} +\frac{1}{r^2\sin\phi}\frac{\partial}{\partial\phi}\bigg(\sin\phi \frac{\partial f}{\partial \phi} \bigg) \\
	\nabla^2 f&= \frac{1}{a^2(\zeta^2+\xi^2)}\left[\frac{\partial}{\partial \zeta}\left((1+\zeta^2)\frac{\partial f}{\partial \zeta}\right) +\frac{\partial}{\partial \xi}\left((1-\xi^2)\frac{\partial f}{\partial \xi}\right) \right] +\frac{1}{a^2(1+\zeta^2)(1-\xi^2)} \frac{\partial^2 f}{\partial \phi^2}.
\end{align}
The \textbf{curl}\index{Curl} can be similarly defined in arbitrary orthogonal curvilinear coordinates as
\begin{align}
	\nabla \times \vec{F} &\equiv \frac{1}{h_1h_2h_3}
	\begin{vmatrix}
		h_1\hat{e}_1 & h_2\hat{e}_2 & h_3\hat{e}_3  \\ 
		\frac{\partial}{\partial e_1} & \frac{\partial}{\partial e_2} & \frac{\partial}{\partial e_3}  \\ 
		h_1F_1 & h_2F_2 & h_3F_3  
	\end{vmatrix} \\
	&= \frac{1}{h_2h_3}\left[\frac{\partial}{\partial u_2}(h_3F_3)-\frac{\partial}{\partial u_3}(h_2F_2)\right]\hat{u}_1+\frac{1}{h_1h_3}\left[\frac{\partial}{\partial u_3}(h_1F_1)-\frac{\partial}{\partial u_1}(h_3F_3)\right]\hat{u}_2 \nonumber\\
	&\hspace{6.72cm}+\frac{1}{h_1h_2}\left[\frac{\partial}{\partial u_1}(h_2F_2)-\frac{\partial}{\partial u_2}(h_1F_1)\right]\hat{u}_3.
\end{align} 

The \textbf{Jacobian}\index{Jacobian} is defined as the determinant of a matrix of partial derivatives \cite{bib:StellarAstrophysics},
\begin{align}
	\frac{\partial(a,b)}{\partial(c,d)} \equiv \begin{vmatrix}
		\left(\frac{\partial a}{\partial c}\right)_d & \left(\frac{\partial a}{\partial d}\right)_c \\
		\left(\frac{\partial b}{\partial c}\right)_d & \left(\frac{\partial b}{\partial d}\right)_c
	\end{vmatrix} =\left(\frac{\partial a}{\partial c}\right)_d\left(\frac{\partial b}{\partial d}\right)_c - \left(\frac{\partial a}{\partial d}\right)_c\left(\frac{\partial b}{\partial c}\right)_d.
\end{align}
By the above definition, we can show the relations,
\begin{align}
	\frac{\partial(b,a)}{\partial(c,d)}=-\frac{\partial(a,b)}{\partial(c,d)} \andspace{1cm} \frac{\partial(a,b)}{\partial(c,d)}=-\frac{\partial(a,b)}{\partial(d,c)}.
\end{align}
It then follows directly that
\begin{align}
	\frac{\partial(a,s)}{\partial(c,s)} = \left(\frac{\partial a}{\partial c}\right)_s \andspace{1cm} \frac{\partial(a,b)}{\partial(a,b)} = 1 \andspace{1cm} \frac{\partial(a,b)}{\partial(c,d)}\frac{\partial(c,d)}{\partial(s,t)} = \frac{\partial(a,b)}{\partial(s,t)}.
\end{align}