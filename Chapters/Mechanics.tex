\chapter{Classical Mechanics}
\thispagestyle{fancy}
\begin{multicols}{2}
Newtons Second Law in Cartesian coordinates and 2D Polar coordinates
\begin{align}
\vec{F}=m\vec{a} = m\boldsymbol{{\ddot{r}}} &\Longleftrightarrow \begin{cases}
F_x = m\ddot{x} \\ F_y = m\ddot{y} \\ F_z = m\ddot{z}
\end{cases}
\\&\Longleftrightarrow \begin{cases}
F_r=m(\ddot{r}-r\dot{\phi}^2) \\
F_\phi=m(r\ddot{\phi}+2\dot{r}\dot{\phi})
\end{cases}
\end{align}
Conservation of energy
\begin{align}
E =\textrm{constant} &=KE+PE \\ &=\frac{1}{2}m|\vec{v}|^2+mgh
\end{align}
Equation of motion for a rocket
\begin{align}
m\dot{v}=-\dot{m}v_{ex}+F^{external}
\end{align}
The center of mass of several particles with a total mass $M$ is
\begin{align}
\vec{R} &=\frac{1}{M}\sum_{\alpha=1}^{n}m_\alpha\vec{r}_\alpha = \frac{m_1\vec{r}_1+\cdots m_n\vec{r}_n}{M} \\
\vec{R} &= \frac{1}{M}\int \vec{r} dm = \frac{1}{M} \int \rho \vec{r} dV 
\end{align}
The mass of an object is defined by the density multiplied by the volume.
\begin{align}
M\equiv \rho V\equiv \iiint_Q\rho(x,y,z) dV
\end{align}
The moment of inertia with respect to a given axis of a solid body with density $\rho(r)$, where $r_\perp$ is the perpendicular distance from the axis of rotation, is defined by the volume integral
\begin{align}
I \equiv \int \rho(\boldsymbol{r}) r_\perp^2 dV \equiv \iiint_{Q}\rho(x,y,z)||\boldsymbol{r}||^2dV 
\end{align}
Angular momentum
\begin{align}
\vec{L}=\vec{r}\times \vec{p}=I\vec{\omega}=I\dot{\theta}
\end{align}
The net external torque is given by
\begin{align}
\vec{\tau}_{ext} = \vec{r} \times \vec{F}= \frac{d\vec{L}}{dt}
\end{align}
The change in kinetic energy as it moves from point a to point b is
\begin{align}
\Delta K &\equiv K_2-K_1= \int_{a}^{b}\vec{F}\cdot d\vec{r} \equiv W(a \rightarrow b) \\
K &=\frac{1}{2}mv^2=\frac{1}{2}I\omega^2=\frac{1}{2}I\dot{\theta}^2
\end{align}
A force $\vec{F}$ on a particle is \textbf{conservative} if (i) it depends only on the particles position, $\vec{F}= \vec{F}(\vec{r})$ and (ii) $\nabla \times \vec{F} = 0$. If $\vec{F}$ is conservative we can define a corresponding \textbf{potential energy} so that
\begin{align}
U(\boldsymbol{r}) &= -W(\boldsymbol{r}_0 \rightarrow \boldsymbol{r}) \equiv \int_{\boldsymbol{r}_0}^{\boldsymbol{r}}\boldsymbol{F}(\boldsymbol{r} ')\cdot d\boldsymbol{r}' \\ 
\vec{F}&=-\nabla \vec{U}
\end{align}
Hooke's Law states that the force needed to extend or compress a spring by some distance is proportional to that distance.
\begin{align}
F=-kx \Longleftrightarrow U=\textrm{constant}+\frac{1}{2}kx^2
\end{align}	
Simple harmonic motion
\begin{align}
\ddot{x}=-\omega^2x \Longleftrightarrow A\cos(\omega t - \delta)
\end{align}
Damped oscillations: If the oscillator is subject to a damping force $-bv$, the
\begin{align}
\ddot{x}+2\beta\dot{x}+\omega_0^2x=0 \textrm{ and } \beta < \omega_0  \nonumber\\ \Longleftrightarrow x(t) =Ae^{-\beta t}\cos(\omega_1t-\delta) \\
\beta = \frac{b}{2m}, \textrm{ }\textrm{ }\textrm{ }
\omega_0 = \sqrt{\frac{k}{m}},\textrm{ }\textrm{ }\textrm{ }
\omega_1 = \sqrt{\omega_0^2-\beta^2}
\end{align}
If the oscillator is also subject to a sinusoidal driving force $F(t)=mf_0\cos(\omega t)$, the long-term motion has the form
\begin{align}
x(t)&=A\cos(\omega t-\delta) \\
A^2 &= \frac{f_0^2}{(\omega_0^2-\omega^2)^2+4\beta^2\omega^2}
\end{align}
\end{multicols}
It is always possible to write a sum of sinusoidal functions as a single sinusoid the form
\begin{align}
f(\theta)=A\cos(\theta)+B\sin(\theta) &\Longleftrightarrow  f(\theta)=C\cos(\theta+\delta) \\
\delta = \arctan(-B/A) \hspace{1cm}&\textrm{and}\hspace{1cm} C = \pm\sqrt{A^2+B^2} \\
f(\theta)=A\cos(\theta)+B\sin(\theta) &\Longleftrightarrow f(\theta)=\textrm{sgn}(A)\sqrt{A^2+B^2}\cos\big(\theta+\arctan(-B/A)\big)
\end{align} 


\begin{multicols}{2}
Any periodic function with period $\tau$ can be written as (A Fourier series with $n \geq 1$)
\begin{align}
f(t)&=\sum_{n=0}^{\infty}[a_n\cos(n\omega t)+b_n\sin(n\omega t)] \\
a_n&= \frac{2}{\tau}\int_{-\tau/2}^{\tau/2}f(t)\cos(n\omega t)dt \\
b_n&= \frac{2}{\tau}\int_{-\tau/2}^{\tau/2}f(t)\sin(n\omega t)dt \\
a_0&= \frac{1}{\tau}\int_{-\tau/2}^{\tau/2}f(t)dt
\end{align}
It is sometimes useful to express the above Fourier series as an exponential
\begin{align}
f(t)&=\sum_{n=-\infty}^{\infty}A_ne^{in\omega t} \\
A_n&=\frac{1}{\tau}\int_{-\tau/2}^{\tau/2}f(t)e^{-in\omega t}dt
\end{align} 
It is important to know $A_n=A^*_{-n}$ so we can write $A_n=\mathfrak{R}(A_n)+i\mathfrak{I}(A_n)$. An important relationship between $A_n$, $a_n$ and $b_n$ then follows as, 
\begin{align}
a_n&=2\mathfrak{R}(A_n)\hspace{0.3cm}\textrm{ and }\hspace{0.3cm} b_n=-2\mathfrak{I}(A_n)
\end{align}
The root-mean square displacement is a good measure of the average response of the oscillator and is given by parseval's theorem
\begin{align}
x_{rms}&=\sqrt{\frac{1}{\tau}\int_{0}^{\tau}x^2dt} \\ &=
\sqrt{A_0^2+\frac{1}{2}\sum_{n=1}^{\infty}A_n^2}
\end{align}
The non-relativistic Lagrangian $\mathcal{L}$ for a conservative system can be defined in terms of the kinetic energy and potential energy of a system as
\begin{align}
\mathcal{L}=KE-PE
\end{align}
An integral of the form
\begin{align}
S=\int_{x_1}^{x_2}f[y(x),y'(x),x]dx
\end{align}
taken along a path $y=y(x)$ is stationary with respect to variations of that path if and only if $y(x)$ satisfies the Euler-Lagrange Equation
\begin{align}
\frac{\partial f}{\partial y}-\frac{d}{dx}\frac{\partial f}{\partial y'}=0.
\end{align}
If there are $n$ dependent variables in the original integral, there are $n$ Euler-Langrange equations. For instance, an integral of the form
\begin{align}
S=\int_{u_1}^{u_2}f[x(u),y(u),x'(u),y'(u),u]du
\end{align}
with two dependent variables [$x(u)$ and $y(u)$], is stationary with respect to variations of $x(u)$ and $y(u)$ if and only if these two functions satisfy the two equations
\begin{align}
\frac{\partial f}{\partial x}&=\frac{d}{du}\frac{\partial f}{\partial x'} \hspace{0.3cm}\textrm{ and }\hspace{0.3cm} \frac{\partial f}{\partial y}=\frac{d}{du}\frac{\partial f}{\partial y'}
\end{align}
For any holonomic system, Newtons second law is equivalent to the $n$ Lagrange equations
\begin{align}
\frac{\partial \mathcal{L}}{\partial q_i}=\frac{d}{dt}\frac{\partial \mathcal{L}}{\partial \dot{q}_i}
\end{align}
The $i$th generalized momentum $p_i$ is defined to be the derivative
\begin{align}
p_i=\frac{\partial \mathcal{L}}{\partial \dot{q}_i}
\end{align}
If $\partial \mathcal{L}/\partial t=0$ then $\mathcal{H}$ is conserved; if the coordinates $q_1,\dots,q_n$ are natural, $\mathcal{H}$ is just the energy of the system. The Hamiltonian $\mathcal{H}$ is defined as
\begin{align}
\mathcal{H}=\sum_{i=1}^{n}p_i\dot{q}_i-\mathcal{L}
\end{align}
The time evolution of a system is given by Hamilton's equations
\begin{align}
\dot{q}_i=\frac{\partial \mathcal{H}}{\partial p_i} \hspace{0.3cm}\textrm{ and }\hspace{0.3cm}\dot{p}_i=-\frac{\partial \mathcal{H}}{\partial q_i}
\end{align}
The Lagrangian for a charge $q$ in an electromagnetic field is
\begin{align}
\mathcal{L}(\boldsymbol{r}, \dot{\boldsymbol{r}}, t)=\frac{1}{2}m\dot{\boldsymbol{r}}^2-q(V-\dot{\boldsymbol{r}}\cdot\boldsymbol{A})
\end{align}
\end{multicols}
