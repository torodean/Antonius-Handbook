\chapter{Quantum Mechanics}
\thispagestyle{fancy}
\begin{multicols}{2}
[Plane waves with electromagnetic wave frequencies and wavelengths]
	\noindent
	\begin{align}
		\psi(x,t) &=A\cos[2\pi(x-ct)/\lambda] \\
		c &= f \lambda \Longleftrightarrow f = \frac{c}{\lambda} \Longleftrightarrow \lambda = \frac{c}{f} \\
		T &=1/f \\
		\psi(x,t) &= A \cos(kx-\omega t) \\
		k &= 2\pi /\lambda \\
		\omega &= 2\pi f = 2 \pi /T
	\end{align}
	The energy in a photon (packet of light)
	\begin{align}
		E &=h f  = \frac{hc}{\lambda} = \hbar \omega \\
		dE &= -\frac{hc}{\lambda^2}d\lambda=-\frac{E^2}{hc}d\lambda \\ |\Delta \lambda | &= hc\frac{\Delta E}{E^2}
	\end{align}
\end{multicols}
The wave equation
\begin{align}
	\frac{1}{v^2}\frac{\partial^2 \psi}{\partial t^2} = \nabla^2\psi
\end{align}
A periodic wave can be constructed from a sum of plane waves
\begin{align}
	\psi(x,t) &= \sum_{i=1}^{n} A_i \cos(k_ix_i-\omega_i t) 
\end{align}
The \textbf{cubit}\index{Cubit} is defined as
\begin{align}
	|\psi\rangle &= c_1|1\rangle + c_0|0\rangle \\
	|\psi\rangle &= c_{11}|11\rangle+c_{01}|01\rangle+c_{10}|10\rangle+c_{00}|00\rangle \\
	&\vdots \nonumber
\end{align}
The quantum mechanical \textbf{expectation value} of an observable $\hat{X}$ in a normalized state $\psi$ is found by integrating over the entire space $\psi^*$ times the result obtained when the corresponding operator  acts on $\psi$. 
\begin{align}
	\langle \psi | \hat{X} | \psi \rangle = \int_{-\infty}^{\infty}\psi^* \hat{X}\psi dx
\end{align}
The state of the system is given by a wavefunction $\psi(\vec{r},t)$.  The probability density is the square modulus of the amplitude
\begin{align}
	P(\vec{r}) d\vec{r} = |\psi(\vec{r})|^2d\vec{r}
\end{align}
The probability of a particle being between $x_1$ and $x_2$ given a normalized wave function $\psi(x,t)$ is
\begin{align}
	P_{x\in x_1:x_2}(t)=\int_{x_1}^{x_2}|\psi(x,t)|^2dx=\int_{x_1}^{x_2} \psi^*(x,t)\psi(x,t) dx
\end{align}
The normalization of a wave function implies the probability over all space is 1.
\begin{align}
	P_{x\in -\infty:\infty}(t) = \int_{-\infty}^{\infty}|\psi(x,t)|^2dx=1
\end{align}
The \textbf{Schr\"{o}dingr Equation}\index{Schr\"{o}dingr Equation} is a partial differential equation that describes how the wavefunction of a physical system evolves over time. The (non-relativistic) Schr\"{o}dingr Equation for a particle moving in a 3-dimensional potential energy field $V(\vec{r})$ is
\begin{align}
	\hat{E}\psi(\vec{r},t) &= \frac{-\hbar^2}{2m} \nabla^2 \psi(\vec{r},t)+V(\vec{r})\psi(\vec{r},t)  \equiv \hat{H}\psi(\vec{r},t)
\end{align}
The general solution to the time-dependent \textbf{Schr\"{o}dingr Equation} is
\begin{align}
	\psi(\vec{r},t) = \sum c_n \psi_n(\vec{r})e^{-iE_n t/\hbar}.
\end{align}
\textbf{The Dirac equation}\index{Dirac equation}: the generalization of the time dependent Schr\"{o}dinger equation for the relativistically correct relationship between energy and momentum. It leads to negative energy states and antiparticles.
\begin{align}
	\bigg[\gamma^0mc^2+\sum_{i=1}^{3}\gamma^i\hat{p}_ic \bigg]\psi(\vec{r},t)=i\hbar\frac{\partial}{\partial t}\psi(\vec{r},t)
\end{align}
Each observable corresponds to a linear operator. A linear operator is something that acts on a state and gives another state.
The Hamiltonian operator is defined as the operator $\hat{H}$ such the energy $E$ of a system with wavefunction $\psi$ is an eigenvalue of $\hat{H}\psi$ or $\hat{H}\psi = E \psi$. 
\begin{align}
	\hat{H} &= \hat{K}+V(\hat{r}) = \frac{\hat{p}^2}{2m}+V(\hat{r})= \frac{-\hbar^2}{2m}\nabla^2	+V(\vec{r})
\end{align}
The Energy operator
\begin{align}
	i\hbar\frac{\partial \psi}{\partial t}=i\hbar\frac{\partial}{\partial t}Ae^{i(kx-\omega t)}=i\hbar(-i\omega)\psi&=\hbar\omega\psi=E\psi \implies\hat{E} = i\hbar \frac{\partial}{\partial t} \\
	\langle\psi| E |\psi \rangle = \int_{-\infty}^{\infty} \psi^* \hat{E} \psi dx &= i\hbar \int_{-\infty}^{\infty} \psi^* \frac{\partial \psi}{\partial t} dx
\end{align}
The operator for a particles kinetic energy is
\begin{align}
	\hat{K}&=\frac{-\hbar^2}{2m}\nabla^2 \\
	\hat{K}\psi = \frac{1}{2m}\hat{p}^2\psi &= \frac{1}{2m}\bigg(-i\hbar\frac{\partial}{\partial x}\bigg)^2\psi = \frac{-\hbar^2}{2m}\frac{\partial^2}{\partial x^2}\psi \\
	\langle\psi| \hat{K}|\psi \rangle = \int_{-\infty}^{\infty}\psi^*\hat{K}\psi dx &= \int_{-\infty}^{\infty}\psi^*\frac{1}{2m}\hat{p}^2\psi dx = \frac{-\hbar^2}{2m}\int_{-\infty}^{\infty}\psi^*\frac{\partial^2}{\partial x^2}\psi dx
\end{align}
The momentum operator
\begin{align}
	\hat{p}&\equiv-i\hbar\nabla = \frac{\hbar}{i}\nabla \\
	\frac{\partial \psi}{\partial x}=\frac{\partial}{\partial x}Ae^{i(kx-\omega t)}=ik\psi&=\frac{ip}{\hbar}\psi \implies\hat{p} = -i\hbar \frac{\partial}{\partial x} \\
	\langle\psi| \hat{p}|\psi \rangle = \int_{-\infty}^{\infty} \psi^* \hat{p} \psi dx &= -i\hbar \int_{-\infty}^{\infty} \psi^* \frac{\partial \psi}{\partial x} dx
\end{align}
The wave function solution for a particle confined to an infinite potential well with walls at $x=0$ and $x=a$ is as follows, with the corresponding energy eigenvalues (with $n\in\mathbb{N}$)
\begin{align}
	\psi(x) &=
	\begin{cases}
		\sqrt{\frac{2}{a}}\sin\left(\frac{n\pi x}{a}\right) &  0\leq x \leq a\\
		0 &  \textrm{ otherwise }
	\end{cases} \\
	E_n &=\frac{\hbar^2\pi^2}{2ma^2}n^2
\end{align}
The solution to The Schr\"{o}dingr Equation for a finite potential well with the potential
\begin{align}
	V(x)=
	\begin{cases}
		\infty & \textrm{ for } x<0 \\
		0 & \textrm{ for } 0 \leq x \leq a \\
		V_1 & \textrm{ for } x>a
	\end{cases}
\end{align}
with $E>V_1$ is
\begin{align}
	\psi(x) &=
	\begin{cases}
		0 & \textrm{ for } x<0 \\
		D\sin(kx) & \textrm{ for } 0 \leq x \leq a \\
		F\cos(k'x)+G\sin(k'x) & \textrm{ for } x>a
	\end{cases} \\
	& \textrm{ with } k'=\sqrt{k^2-\frac{2mV_1}{\hbar^2}}
\end{align}
with $E<V_1$ is
\begin{align}
	\psi(x) &=
	\begin{cases}
		0 & \textrm{ for } x<0 \\
		D\sin(kx) & \textrm{ for } 0 \leq x \leq a \\
		Fe^{-\gamma x} & \textrm{ for } x>a
	\end{cases} \\
	& \textrm{ with } \gamma^2=\frac{2m(V_1-E)}{\hbar^2}=\frac{2mV_1}{\hbar^2}-k^2
\end{align}
The solutions for a finite barrier (with the probability of reflection as $R=|B|^2/|A|^2$ and the probability of transmission as $T=|F|^2/|A|^2$) are
\begin{align}
	\psi_1&=Ae^{ikx}+Be^{-ikx} \hspace{1cm}\textrm{(incident + reflected)} \\
	\psi_2&=Ce^{ik'x}+De^{-ik'x} \hspace{1cm}\textrm{(intermediate)}\\
	\psi_3&=Fe^{ikx} \hspace{1cm}\textrm{(transmitted)} \\
	k&=\sqrt{2mE/\hbar^2}\\
	k'&=\sqrt{2m(E-U_0)/\hbar^2}
\end{align}
For any two Hermitian operators $A$ and $B$,
\begin{align}
	\Delta A \Delta B \geq \frac{1}{2}| \langle i [A, B] \rangle |
\end{align}
Atomic quantum numbers
\begin{align}
	n &= \textrm{ Principle Quantum Number } [n \in \mathbb{N}]\\
	\ell &= \textrm{ Orbital Angular Momentum Quantum Number } [\ell \in \mathbb{N}\union\{0\}, \ell < n] \\
	m_\ell &= \textrm{ Magnetic Quantum Number }[m_\ell \in [-\ell, \ell], m_\ell \in \mathbb{Z}]
\end{align}

For a simple harmonic oscillator\index{Harmonic oscillator}, the energies and eigenfunctions are given by,
\begin{align}
	V(x)&=\frac{1}{2}kx^2=\frac{1}{2}m\omega^2x^2\implies k=m\omega^2 \Longleftrightarrow \omega=\sqrt{\frac{k}{m}} \\
	E_n&=\left(n+\frac{1}{2}\right)\hbar\omega=\left(n+\frac{1}{2}\right)\hbar\sqrt{\frac{k}{m}}=\left(n+\frac{1}{2}\right)\frac{\hbar}{x}\sqrt{\frac{2V(x)}{m}} \\
	H | n \rangle &= E_n |n \rangle, \hspace{1cm} H = \hbar \omega \left(a^\dagger a + \frac{1}{2}\right), \hspace{1cm}[a,a^\dagger]=1 \\	
	|n\rangle &= \frac{(a^\dagger)^n|0\rangle}{\sqrt{n!}}, \hspace{1cm} a^\dagger|n\rangle = \sqrt{n+1}|n+1\rangle, \hspace{1cm}a|n \rangle = \sqrt{n}|n-1\rangle
\end{align}
The normalized angular wave functions are called \textbf{spherical harmonics}\index{Spherical harmonics} and are given by
\begin{align}
	Y_\ell^m(\theta,\phi) = \epsilon\sqrt{\frac{(2\ell+1)}{4\pi}\frac{(\ell-|m|)!}{(\ell+|m|)!}}e^{im\phi}P_\ell^m(\cos\theta), \hspace{1cm} \textrm{with} \hspace{1cm}\epsilon = \begin{cases} (-1)^m & m \geq 0 \\ 1 & m < 0 \end{cases}.
\end{align}
The \textbf{spherical harmonics} are automatically orthogonal so,
\begin{align}
	\int_0^{2\pi}\int_0^\pi[Y_\ell^m(\theta,\phi)]^*[Y_{\ell'}^{m'}(\theta,\phi)]\sin\theta d\theta d\phi=\delta_{\ell \ell'}\delta_{m m'}
\end{align}
The normalized \textbf{hydrogen wave functions} containing the quantum numbers $n, m,$ and $\ell$ are
\begin{align}
	\Psi_{n\ell m}(r,\theta,\phi) = \sqrt{\left(\frac{2}{na}\right)^3\frac{(n-\ell-1)!}{2n[(n+\ell)!]^3}}e^{-r/na}\left(\frac{2r}{na}\right)^\ell\left[L_{n-\ell-1}^{2\ell+1}(2r/na)\right]Y_{\ell}^{m}(\theta,\phi).
\end{align}
The ground state wave functions and energy of Hydrogen is
\begin{align}
	\Psi_{100}(r,\theta,\phi) = \frac{1}{\sqrt{\pi a^3}}e^{-r/a} \andspace{1cm} E_1 = -\left[\frac{m}{2\hbar^2}\left(\frac{e^2}{4\pi\epsilon_0}\right)^2\right] = -13.6 eV
\end{align}
Each operator $\hat{Y}$ has a set of eigenvalues y which are the possible values you can get on doing a measurement of Y. Each eigenvalues y is associated with an eigenstate $\phi_y(x)$ which is the state for which the values of Y is exactly y with no uncertainty. You can find the eigenstates and eigenvalues of an operator by 
\begin{align}
	\hat{Y}\phi_y(x) = y \phi_y(x).
\end{align}
The eigenstates of any operator $\hat{Y}$ form a complete orthonormal basis of states so we can write any state $\psi(x)$ in terms of them.
\begin{align}
	\psi(x) \equiv \sum_{y}A_y\phi_y(x) \hspace{0.5cm}\textrm{or}\hspace{0.5cm} \psi(x) \equiv \int A(y)\phi_y(x)dy.
\end{align}
To solve for the coefficients in the above expression we can use Fouriers trick, or
\begin{align}
	A(y) = \int \phi_y^*(x)\psi(x)dx 	
\end{align}
If you are within operator space you can find the expectation value of an operator by
\begin{align}
	\langle \hat{y} \rangle = \int y |A(y)|^2 dy.
\end{align}
The commutation relation is a relationship between two operators and given by
\begin{align}
	[x,y] \equiv xy-yx &\implies [x,y] = -[y,x] \\
	[x,y]=[y,x]=0 &\implies \textrm{$x$ and $y$ commute} \\
	[xy,z] = x[y,z]+[x,z]y \hspace{1cm} &\textrm{and} \hspace{1cm}[x,yz] = y[x,z]+[x,y]z.adv
\end{align} 
Position and momentum are related via commutation by the following:
\begin{align}
	[x,p_x]=[y,p_y]=[z,p_z]=i\hbar \\
	[x,p_y]=[x,p_z]=[y,p_x]=[y,p_z]=[z,p_x]=[z,p_y]=0.
\end{align} 
The angular momentum operators are related via commutation by the following:
\begin{align}
	[L_x,L_y]=i\hbar L_z \andspace{1cm} &[L_y,L_z]=i\hbar L_x \andspace{1cm} [L_z,L_x]=i\hbar L_y \\
	[L_x,\vec{r}]=i\hbar (z-y) \andspace{1cm} &[L_x,\vec{L}] = i\hbar (L_z-L_y) \andspace{1cm} [L_x,\vec{p}] = i\hbar(p_z-p_y) \\
	[\vec{L}^2,L_\pm]=0 \andspace{1cm}&[\vec{L}^2,L_z] = 0 \andspace{1cm} [L_z,L_\pm]=\pm \hbar L_\pm
\end{align}
The angular momentum operators as well as the raising and lowering operators are given by
\begin{align}
	\vec{L} &= \frac{\hbar}{i}\left(\hat{\phi}\frac{\partial}{\partial \theta}- \frac{\hat{\theta}}{\sin \theta}\frac{\partial}{\partial \phi}\right) \hspace{1cm} \textrm{and} \hspace{1cm} L_z =\frac{\hbar}{i} \frac{\partial}{\partial \phi}\\
	L_x &= \frac{L_++L_-}{2}= \frac{\hbar}{i}\left(-\sin\phi \frac{\partial}{\partial \theta}-\cot\theta \cos \phi \frac{\partial}{\partial \phi}\right) \\  L_y &= \frac{L_+-L_-}{2i}= \frac{\hbar}{i}\left(\cos\phi \frac{\partial}{\partial \theta}-\cot\theta \sin \phi \frac{\partial}{\partial \phi}\right) \\
	L_{\pm} &= L_x\pm i L_y = \pm \hbar e^{\pm i \phi} \left(\frac{\partial}{\partial \theta}\pm  i \cot(\theta)\frac{\partial}{\partial \phi}\right) \\
	\vec{L}^2 &= L_x+L_y+L_z =-\hbar^2 \left[\frac{1}{\sin\theta}\frac{\partial}{\partial \theta}\left(\sin\theta\frac{\partial }{\partial \theta}\right)+\frac{1}{\sin^2\theta}\frac{\partial^2}{\partial \phi^2}\right]
\end{align}
The angular momentum operators satisfy
\begin{align}
	\vec{L}^2 | \ell, m \rangle = \hbar^2\ell(\ell+1)| \ell, m \rangle \andspace{1cm} L_z| \ell, m \rangle = \hbar m | \ell, m \rangle \\
	L_{\pm} | \ell, m \rangle = \hbar \sqrt{(\ell \mp m)(\ell \pm m +1)} | \ell, m \pm 1 \rangle
\end{align}
The fundamental commutation relations for spin are
\begin{align}
[S_x,S_y]=i\hbar S_z \andspace{1cm} [S_y,S_z]=i\hbar S_x \andspace{1cm} [S_z,S_x]=i\hbar S_y.
\end{align}
The general state of a spin-1/2 particle can be expressed as a two element column matrix (or spinor):
\begin{align}
\chi= \begin{pmatrix}
a\\b
\end{pmatrix} = a \begin{pmatrix}
1\\0
\end{pmatrix}+b\begin{pmatrix}
0\\1
\end{pmatrix}= a \chi_++b\chi_-.
\end{align}
The spin matrices are given by
\begin{align}
\vec{S}_x &= \frac{\hbar}{2}\begin{pmatrix}
0 & 1 \\ 1 & 0
\end{pmatrix}, \hspace{1cm}\vec{S}_y = \frac{\hbar}{2}\begin{pmatrix}
0 & -i \\ i & 0
\end{pmatrix}, \hspace{1cm}\vec{S}_z = \frac{\hbar}{2}\begin{pmatrix}
1 & 0 \\ 0 & -1
\end{pmatrix}, \\ \vec{S}_+ &= \hbar\begin{pmatrix}
0 & 1 \\ 0 & 0
\end{pmatrix}, \hspace{1cm}\vec{S}_- = \hbar\begin{pmatrix}
0 & 0 \\ 1 & 0
\end{pmatrix}, \hspace{1.3cm}\vec{S}^2 = \frac{3\hbar^2}{4}\begin{pmatrix}
1 & 0 \\ 0 & 1
\end{pmatrix}
\end{align}
The \textbf{Pauli Spin Matrices}\index{Pauli Spin Matrices} are then given by
\begin{align}
\sigma_x \equiv \begin{pmatrix}
0 & 1 \\ 1 & 0
\end{pmatrix}, \hspace{1cm}\sigma_y \equiv \begin{pmatrix}
0 & -i \\ i & 0
\end{pmatrix}, \hspace{1cm}\sigma_z \equiv \begin{pmatrix}
1 & 0 \\ 0 & -1
\end{pmatrix}
\end{align}
The combined state $| s, m \rangle$  with total spin s and z-component m will be some linear combination of the composite states $| s_1, m_1 \rangle$ and $| s_2, m_2 \rangle$ and depends on the Clebsch-Gordan\index{Clebsch-Gordan} Coefficients $C_{m_1m_2m}^{s_1s_2s}$ (page \pageref{Clebsch-Gordan}):
\begin{align}
| s, m \rangle = \sum_{m=m_1+m_2} C_{m_1m_2m}^{s_1s_2s}| s_1, m_1 \rangle| s_2, m_2 \rangle
\end{align}

