\chapter{Complex Analysis}
\thispagestyle{fancy}
\section{Complex Numbers}
\begin{multicols}{2}
The set of complex numbers is defined such that
\begin{align}
\mathbb{C}=\{a+bi : a,b \in \mathbb{R}\}.
\end{align}
A complex number can be defined by its real part and it's imaginary part
\begin{align}
i^2 = -1 &\Longleftrightarrow i=\sqrt{-1} \Longleftrightarrow \frac{1}{i}=-i\\
z = x+iy &\Longleftrightarrow z^*=x-iy 
\end{align}
We can express the real and imaginary parts of a complex number in terms of the number and its complex conjugate
\begin{align}
\mathfrak{R}(z)&= \frac{1}{2}(z+z^*) \\
\mathfrak{I}(z)&= \frac{1}{2}i(z-z^*) 
\end{align}
Just like a two-dimensional vector, a complex number has the magnitude $|z|$ as well as an angle $\theta$ with respect to the horizontal axis of the complex plane.
\begin{align}
|z|^2 &= z^* z=x^2+y^2=|z|e^{-i\theta}|z|e^{i\theta} \\
\tan(\theta) &= \frac{\mathfrak{I}(z)}{\mathfrak{R}(z)}  = \frac{y}{x} = \frac{i(z-z^*)}{(z+z^*)}
\end{align}
A complex number can thus be expressed in terms of magnitude and the phase angle
\begin{align}
z=|z|(\cos(\theta)+i\sin(\theta))
\end{align}
Euler's Identity\index{Euler's Identity}/relation
\begin{align}
e^{i\theta}&=cos(\theta)+i\sin(\theta)
\end{align}
With the aid of Eulers identity, we can write any complex number as
\begin{align}
z&=|z|e^{i\theta} \\
z^n&=|z|^ne^{in\theta}
\end{align}
A useful property of conjugates is
\begin{align}
	a^*+b^*=(a+b)^*
\end{align}
\end{multicols}
Powers and roots of a complex number can be determined from the exponential form of a complex number
\begin{align}
	z^n &= (re^{i\theta})^n = r^n e^{in\theta} \\
	(e^{i\theta})^n &= e^{in\theta} = (\cos\theta+i\sin\theta)^n = \cos (n\theta) + i\sin (n\theta) \\
	z^{1/n} &= (re^{i\theta})^{1/n} = r^{1/n} e^{i\theta/n} = \sqrt[n]{r}\left(\cos\frac{\theta}{n}+i\sin\frac{\theta}{n}\right)
\end{align}
Much like in trigonometry, we can define complex numbers using trigonometric identities:
\begin{align}
	\sin z = \frac{e^{iz}-e^{-iz}}{2i},\hspace{1cm}\cos z = \frac{e^{iz}+e^{-iz}}{2},\hspace{1cm}\sinh z = \frac{e^{z}-e^{-z}}{2},\hspace{1cm}\cosh z = \frac{e^{z}+e^{-z}}{2}
\end{align}
The logarithm of a complex number can be manipulated as a normal log with
\begin{align}
	\ln (z) = \ln(re^{i\theta}) = \ln(r)+\ln(e^{i\theta}) = \ln(r)+i\theta 
\end{align}
A few Trigonometric identities follow as:
\begin{align}
	\arcsin z &= -i \ln (iz\pm \sqrt{1-z^2})\\ \arccos z &= i \ln (z\pm \sqrt{z^2-1}) \\ \arctan z &= \frac{1}{2i} \ln \left(\frac{1+iz}{1-iz}\right)
\end{align}

\section{Complex Functions}
A complex function of $z$ can be expressed in terms of two real functions $u(x,y)$ and $v(x,y)$,
\begin{align}
	f(z) = f(x+iy) = u(x,y)+iv(x,y).
\end{align}
The derivative of $f(z)$ is defined by
\begin{align}
	f'(z) = \frac{df}{dz} = \lim\limits_{\Delta z \rightarrow 0}\frac{\Delta f}{\Delta z} =\lim\limits_{\Delta z \rightarrow 0} \frac{f(z+\Delta z)-f(z)}{\Delta z}=\lim\limits_{\Delta x, \Delta y \rightarrow 0} \frac{f(z+\Delta x+i\Delta y)-f(z)}{\Delta x+i\Delta y}.
\end{align}
\begin{defn}[Analytic Function \cite{bib:Methods_Of_Theoretical}]{AnalyticFunc}
A function $f(z)$ is \textbf{analytic} (or regular or holomorphic or mono-genic) in a region$^∗$ of the complex plane if it has a (unique) derivative at every point of the region. The statement “$f(z)$ is analytic at a point $z = a$” means that $f(z)$ has a derivative at every point inside some small circle about $z = a$.
\end{defn}
The \textbf{Cauchy-Riemann conditions}\index{Cauchy-Riemann conditions} state that if $f(z) = u(x, y) + iv(x, y)$ is analytic in a region, then in that region
\begin{align}
	\frac{\partial u}{\partial x} = \frac{\partial v}{\partial y}, \hspace{1cm}\textrm{and}\hspace{1cm} \frac{\partial v}{\partial x} = - \frac{\partial u}{\partial y}.
\end{align}
From this we also have
\begin{align}
	\frac{\partial f}{\partial x} =\frac{\partial u}{\partial x} + i\frac{\partial v}{\partial x}, \hspace{1cm}\textrm{and}\hspace{1cm} \frac{\partial f}{\partial y} =\frac{\partial u}{\partial y} + i\frac{\partial v}{\partial y}.
\end{align}
\begin{defn}[Regular and Singular Points\index{Singular Points} \cite{bib:Methods_Of_Theoretical}]{singularPoints}
	A \textbf{regular point} of $f(z)$ is a point at which $f (z)$ is analytic.
	A \textbf{singular point} or singularity of $f (z)$ is a point at which $f (z)$ is not analytic.
	It is called an isolated singular point if $f (z)$ is analytic everywhere else inside
	some small circle about the singular point.
\end{defn}
A few useful theorems \cite{bib:Methods_Of_Theoretical} include 
	\begin{enumerate}
	\item If $f(z)$ is analytic in a region, then it has derivatives of all orders at points inside the region and can be expanded in a Taylor series about any point $z_0$ inside the region.
	The power series converges inside the circle about $z_0$ that extends to the nearest
	singular point. 
	
	\item If $f(z) = u + iv$ is analytic in a region, then $u$ and $v$ satisfy Laplace's equation in the region (that is, u and v are harmonic functions).
		
	\item Any function $u$ (or $v$) satisfying Laplace's equation in a simply-connected region, is the real or imaginary part of an analytic function $f(z)$.
	\end{enumerate}
	Given a complex function $f(z)=u(x,y)+iv(x,y)$, we can take the integral along a path $\ell$ using
	\begin{align}
		\int_\ell f(z) dz = \int_\ell f(x+iy) (dx+idy) = \int_\ell u(x,y)dx - v(x,y)dy + i\int_\ell u(x,y)dy + v(x,y)dx.
	\end{align}
\begin{fancybox}[Cauchy's Theorem\index{Cauchy's!Theorem} \cite{bib:Methods_Of_Theoretical}]{}
Let C be a simple closed curve (one which does not cross itself) with a continuously turning tangent except possibly at a finite number of points (that is, we allow a finite number of corners, but otherwise the curve must be “smooth”). If $f(z)$ is analytic on and inside $C$, then
\begin{align}
	\oint_C f(z) dz =0
\end{align}
\end{fancybox}

\begin{fancybox}[Cauchy's Integral Formula\index{Cauchy's!Integral Formula} \cite{bib:Methods_Of_Theoretical}]{}
	If $f(z)$ is analytic on and inside a simple closed curve $C$, the value of $f (z)$ at a point $z = a$ inside $C$ is given by the following contour integral along $C$:
	\begin{align}
		f(a) = \frac{1}{2\pi i}\oint_C \frac{f(z)}{z-a}dz	
	\end{align}
\end{fancybox}









\begin{fancybox}[Laurent's Theorem \cite{bib:Methods_Of_Theoretical}, \cite{bib:Wolfram}]{}
	Let $C_1$ and $C_2$ be two circles with center at $z=z_0$ with radii $r_1$ and $r_2 < r_1$ respectively. Let $f(z)$ be analytic in the region $R$ between the circles. Then $f(z)$ can be expanded in a series of the form
	\begin{align}
		f(z) &= a_0+a_1(z-z_0)+a_2(z-z_0)^2+\cdots + \frac{b_1}{z-z_0}+\frac{b_2}{(z-z_0)^2}+\cdots \\&= \sum_{k=0}^{\infty}a_k(z-z_0)^k+\sum_{k=1}^{\infty}b_k(z-z_0)^{-k},
	\end{align}
	convergent in $R$. Such a series is called a Laurent series. The “b” series is called the principal part of the Laurent series. The coefficients have the solutions 
	\begin{align}
		a_k=\frac{1}{2\pi i} \oint_{C_1}\frac{f(z')dz'}{(z' -z_0)^{k+1}},\hspace{1cm}\textrm{and}\hspace{1cm}b_k=\frac{1}{2\pi i} \oint_{C_2}(z' -z_0)^{k-1}f(z')dz'
	\end{align}
\end{fancybox}

\begin{defn}[Poles, Residue and Singularities \cite{bib:Methods_Of_Theoretical}]{singularPoints}
	\begin{enumerate}
		\item If all the b's are zero, $f (z)$ is analytic at $z = z_0$ , and we call $z_0$ a \textbf{regular point}.
		
		\item If $b_n \neq 0$, but all the b's after $b_n$ are zero, $f(z)$ is said to have a \textbf{pole} of order $n$ at $z = z_0$ . If $n = 1$, we say that $f(z)$ has a simple pole.
		
		\item If there are an infinite number of b's different from zero, $f(z)$ has an \textbf{essential
		singularity} at $z = z_0$.
		
		\item The coefficient $b_1$ of $1/(z-z_0)$ is called the \textbf{residue} of $f(z)$ at $z = z_0$.
	\end{enumerate}
\end{defn}
	
