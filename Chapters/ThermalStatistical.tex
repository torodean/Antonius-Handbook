\chapter{Thermal \& Statistical Physics}
\thispagestyle{fancy}

\begin{multicols}{2}
	\section{States of a Model System}
	The multiplicity function\index{Multiplicity function} for a system of N magnets with a spin excess $2s=	N_{\uparrow}-N_{\downarrow}$ is
	\begin{align}
		g(N,s)=\frac{N!}{(\frac{N}{2}+s)!(\frac{N}{2}-s)!} = \frac{N!}{N_{\uparrow}!N_{\downarrow}!}.
	\end{align}
	It is often useful to evaluate $g(N,s)$ within a logarithm in which the \textbf{Stirling approximation}\index{Stirling approximation} becomes useful.
	\begin{align}
		N! \approx N^N\sqrt{2\pi N} \exp\left(-N+\frac{1}{12N}+\cdots\right).
	\end{align}
	It is often useful to take the logarithm of this which gives
	\begin{align}
		\log N!\cong \frac{\log 2\pi}{2}+\left(N+\frac{1}{2}\right)
		\log N-N.
	\end{align}
	In the limit $s/N << 1$, with $N>>1$, we have the Gaussian approximation 
	\begin{align}
		g(N,s) &\cong g(N,0)\exp\left(\frac{-2s^2}{N}\right) \\
		g(N,0)&\simeq 2^N\sqrt{\frac{2}{\pi N}}.
	\end{align}
	The exact value of $g(N,0)$ is given by
	\begin{align}
		g(N,0) = \frac{N!}{(N/2)!(N/2)!}.
	\end{align}
	The average value, or mean value, of a function f(s) taken over a probability distribution P(s) is defined as
	\begin{align}
		\langle f \rangle &=\sum_{s} f(s)P(s), \\
		1 &= \sum_{s} P(s).
	\end{align}
	The binomial distribution has the property 
	\begin{align}
		\sum_{s}g(N,s)=2^N.
	\end{align}
	If all states of the model spin system are equally likely, the average value of $s^2$ is
	\begin{align}
		\langle s^2 \rangle = \frac{\int_{-\infty}^{\infty}s^2 g(N,s) ds }{\int_{-\infty}^{\infty} g(N,s) ds } = \frac{N}{4}
	\end{align}
	The energy interaction of a single magnetic moment\index{Magnetic moment} $\vec{m}$ with a fixed external magnetic field $\vec{B}$ is
	\begin{align}
		U=-\vec{m}\cdot \vec{B}.
	\end{align}
	For a model system of N elementary magnets, each with two allowed orientations in a uniform magnetic field $\vec{B}$, the total potential energy U is
	\begin{align}
		U &=\sum_{i=0}^{N}U_i=-\vec{B}\cdot \sum_{i=0}^{N}m_i \\
		&=-2smB = -MB.
	\end{align}
	\section{Entropy And Temperature}
	If $P(s)$ is the probability that a system is in the state $X$, the average value of a quantity $X$ is 
	\begin{align}
		\langle X \rangle = \sum_{s}X(s)P(s).
	\end{align}
	The number of combined systems 1 and 2 (with $s=s_1+s_2$) is
	\begin{align}
		g(s) = \sum_s g_1(s_1)g_2(s-s_1).
	\end{align}
	The relation $s=k_B\sigma$ connects the conventional entropy S with the fundamental entropy $\sigma$. The \textbf{entropy}\index{Entropy} $\sigma(N,U)$ is given by 
	\begin{align}
		\sigma(N,U) = \log g(N,U).
	\end{align}
	The fundamental temperature $\tau$ is defined by the relation
	\begin{align}
		\frac{1}{\tau} = \left(\frac{\partial \sigma}{\partial U}\right)_{N,V}.
	\end{align}
	\section{Boltzmann Distribution and Helmholtz Free Energy}
	The \textbf{partition function}\index{Partition function} Z is
	\begin{align}
		Z \equiv \sum_{s} \exp\left(-\frac{\epsilon_s}{\tau}\right).
	\end{align}
	The probability of finding a system of N particles in a state s of energy $\epsilon_s$ when the system is in thermal contact with a large reservoir at temperature $\tau$ is
	\begin{align}
		P(\epsilon_s) = \frac{1}{Z}\exp\left(-\frac{\epsilon_s}{\tau}\right).
	\end{align} 
	The pressure is given by
	\begin{align}
		P = -\left(\frac{\partial U}{\partial V}\right)_\sigma = \tau\left(\frac{\partial \sigma}{\partial V}\right)_U.
	\end{align}
	The \textbf{Helmholtz Free Energy}\index{Helmholtz Free Energy} is a minimum in equilibrium for a system held at constant $\tau, V$ and is defined as
	\begin{align}
		F\equiv U-\tau \sigma.	
	\end{align}
	From this we have
	\begin{align}
		\sigma &= -\left(\frac{\partial F}{\partial \tau}\right)_V \\
		P &= -\left(\frac{\partial F}{\partial V}\right)_\tau.
	\end{align}
	For an ideal monotonic gas of N atoms of spin zero with $n=N/V << n_Q$,
	\begin{align}
		Z_n = \frac{Z_1^N}{N!}=\frac{(n_Q V)^N}{N!},
	\end{align}
	The quantum concentration $n_Q$ is defined by
	\begin{align}
		n_Q \equiv \left(\frac{M\tau}{2\pi\hbar^2}\right)^{3/2}.
	\end{align}
	Furthermore, we have
	\begin{align}
		PV&=N\tau \\
		\sigma &=N\left[\log\left(\frac{n_Q}{n}\right)+\frac{5}{2}\right] \\
		C_V&=\frac{3}{2}N.  
	\end{align}
	The thermal average energy of an atom a a box is
	\begin{align}
		U = \langle \epsilon \rangle = \tau^2 \frac{\partial \log (Z_1)}{\partial \tau}.
	\end{align}
	For a system of fixed volume in thermal contact with a reservoir, the mean square fluctuation in energy of the system is
	\begin{align}
		\langle(\epsilon - 	\langle\epsilon \rangle)^2 \rangle = \tau^2\left(\frac{\partial U}{\partial \tau}\right)_V
	\end{align}  
	\section{Thermal Radiation and Planck Distribution}
	The \textbf{Planck distribution function}\index{Planck distribution function} for the thermal average number of photons in a cavity mode of frequency $\omega$ is
	\begin{align}
		\langle s \rangle = \frac{1}{\exp(\hbar \omega/t)-1}.
	\end{align}
	The \textbf{Stefan-Boltzmann law}\index{Stefan-Boltzmann law} for the radiant energy density in a cavity at temperature $\tau$ is
	\begin{align}
		\frac{U}{V} = \frac{\pi^2}{15\hbar^3 c^3}\tau^4.	
	\end{align}
	The \textbf{Planck radiation law}\index{Planck radiation law} for the energy per unit volume per unit range of frequency is
	\begin{align}
		u_\omega = \frac{\hbar}{\pi^2c^3}\frac{\omega^3}{\exp(\hbar \omega/t)-1}.
	\end{align}
	The flux density of radiant energy $J_\nu$ and the Stefan-Boltzmann constant are
	\begin{align}
		J_\nu &= \sigma_BT^4 \\
		\sigma_B &= \frac{\pi^2k_B^4}{60\hbar^3c^3}.
	\end{align}
	The Debye\index{Debye} low temperature limit of the heat capacity of a dielectric solid (where $\vartheta$ is the Debye temperature) is, in conventional units,
	\begin{align}
		C_V &= \frac{12\pi^4Nk_B}{5}\left(\frac{T}{\vartheta}\right)^3 \\
		\vartheta &= \frac{\hbar c}{k_B}\left(\frac{6\pi^2N}{V}\right)^{1/3}.
	\end{align}
\end{multicols}