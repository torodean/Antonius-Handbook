\chapter{C++}
\thispagestyle{fancy}
\lstset{language=C++, style=cpp}

\section{Basic Input and Output\index{Input and Output}}
To output text via a terminal you can use:
\begin{lstlisting}
std::string text = "Hello World!";
std::cout << text << std::endl; //std::endl is equivalent to the new-line character.
\end{lstlisting}

To get input as a user in the type of a std::string, you can use:
\begin{lstlisting}
std::string input = "";
std::cout << "Enter some text: ";
std::getline(std::cin, input);
\end{lstlisting}

\subsection{Simulate Key Strokes (Windows Only)\index{Simulate Key Strokes}}

First the correct files must be included and an event must be setup.
\begin{lstlisting}
#define WINVER 0x0500
#include <windows.h> 

INPUT ip;

ip.type = INPUT_KEYBOARD; // Set up a generic keyboard event.    
ip.ki.wScan = 0; // hardware scan code for key                                   
ip.ki.time = 0;
ip.ki.dwExtraInfo = 0;
\end{lstlisting}

After this, functions can be setup to simulate various keys based on the specific key codes, two examples of such are
\begin{lstlisting}
void space(){ 
	// Press the "space" key.  
	ip.ki.wVk = VK_SPACE; // virtual-key code for the "space" key.                                                                                
	ip.ki.dwFlags = 0; // 0 for key press                                                                                                        
	SendInput(1, &ip, sizeof(INPUT));
	
	// Release the "space" key                                                                                                                   
	ip.ki.wVk = VK_SPACE; // virtual-key code for the "space" key.                                                                                
	ip.ki.dwFlags = KEYEVENTF_KEYUP; // KEYEVENTF_KEYUP for key release                                                                          
	SendInput(1, &ip, sizeof(INPUT));
	Sleep(50);
}

void one(){ 
	// Press the "1" key.    
	ip.ki.wVk = 0x31; // virtual-key code for the "1" key.                                          
	ip.ki.dwFlags = 0; // 0 for key press                                                                                                        
	SendInput(1, &ip, sizeof(INPUT));
	
	// Release the "1" key.                                                                                                                    
	ip.ki.wVk = 0x31; // virtual-key code for the "1" key.                                                                                      
	ip.ki.dwFlags = KEYEVENTF_KEYUP; // KEYEVENTF_KEYUP for key release.                                                                          
	SendInput(1, \&ip, sizeof(INPUT));
	Sleep(50);
}
\end{lstlisting}

A similar method can be used to simulate mouse clicks. And example for left click follows
\begin{lstlisting}
void leftclick(){
	INPUT ip={0};
	// left down                                                                                                                                   
	ip.type = INPUT_MOUSE;
	ip.mi.dwFlags = MOUSEEVENTF_LEFTDOWN;
	SendInput(1,&Input,sizeof(INPUT));
	
	// left up                                                                                                                                     
	ZeroMemory(&Input,sizeof(INPUT));
	ip.type = INPUT_MOUSE;
	ip.mi.dwFlags = MOUSEEVENTF_LEFTUP;
	SendInput(1,&Input,sizeof(INPUT));
	Sleep(50);
}
\end{lstlisting}


\section{Variable Types\index{Variable Types}}

Creating and using a vector\index{Vector}.
\begin{lstlisting}
#include <vector>

int size1 = 5;
int size2 = 6;

//Creates a vector named V1 containing int's with a size of 5 and sets each element to 0. 
std::vector<int> V1(size1); 

//Creates a 2-D vector (vector containing vectors) of size 5x6 named V2 containing doubles;
std::vector< std::vector<double>> V2(size1, std::vector<double>(size2)); 

V1[0] = 8; //Sets the first element in V1 to 8.

V2[0][3] = 3.1415; //Sets the 4th element in the first row of V2 to 3.1415.

\end{lstlisting}


\subsection{Converting Between Types\index{Converting Between Types}}

\subsection*{std::string to int\index{std::string to int}}
To convert a string to an integer you can use:
\begin{lstlisting}
std::string text = "31415";
int number = std::stoi(text);
\end{lstlisting}

\section{Mathematical Commands}

\subsection*{Prime Number\index{Prime Number}}
A simple brute for method to determines if a number of type long is prime or not.
\begin{lstlisting}
bool isPrime(long num) {  
	int c = 0;   //c is a counter for how many numbers can divide evenly into num
	if (num == 0 || num == 1 || num == 4) {
		return false;
	}
	for (long i = 1; i <= ((num + 1) / 2); i++) {
		if (c < 2) {
			if (num % i == 0) {
				c++;
			}
		} else {
			return false;
		}
	}
	return true;
}
\end{lstlisting}

\section{System Commands\index{System Commands}}

\subsection*{Sleep\index{Sleep}}
Make the thread sleep for some amount of time using the std::chrono to determine the duration \cite{cpp:chrono}.
\begin{lstlisting}
#include <thread>
#include <chrono>

std::this_thread::sleep_for(std::chrono::milliseconds(50)); //Makes the system sleep for 50 milliseconds.

std::this_thread::sleep_for(std::chrono::seconds(50)); //Makes the system sleep for 50 seconds.
\end{lstlisting}

On a Windows specific program this can be simplified by including the windows.h header
\begin{lstlisting}
#include <windows.h>

Sleep(50); //Makes the system sleep for 50 milliseconds.

Sleep(5000); //Makes the system sleep for 50 seconds.
\end{lstlisting}

