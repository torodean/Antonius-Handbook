\chapter{Linux}
\thispagestyle{fancy}
\lstset{language=Bash, style=bash}

\section{System Related Commands\index{System Related Commands}}

Retreive information and valid arguments for a command.
\begin{lstlisting}
COMMAND --help   # COMMAND must be a valid command such as cd, ls, etc...
\end{lstlisting}

List information about File(s) (in the current directory by default).
\begin{lstlisting}
ls        # list all items in a directory
ls -1     # list all items in a directory (one item per line)
ls -lh    # list all items in a directory with size, owner, and date modified
\end{lstlisting}

Changing directory via terminal
\begin{lstlisting}
cd /directory # Changes the directory to the subdirectory /directory
cd ..         # Goes back one directory
\end{lstlisting}

Copy a file or directory to a different linux computer
\begin{lstlisting}
#To copy a file.
scp <File Path> username@computer:"<path to copy to>"

#To copy a directory.
scp -r <File Path> username@computer:"<path to copy to>"
\end{lstlisting}

Show information about the file system on which each FILE resides, or all file systems by default.
\begin{lstlisting}
df 
\end{lstlisting}

How to display the processes that are currently running.
\begin{lstlisting}
ps aux
\end{lstlisting}

To search the results of a command for a string of characters one can use the grep command. For example:
\begin{lstlisting}
ps aux | grep "firefox"
\end{lstlisting}

Restore power/battery icon if it disappears.
\begin{lstlisting}
/usr/lib/x86_64-linux-gnu/indicator-power/indicator-power-service &disown 
\end{lstlisting}

Restore volume icon/control button if it disappears.
\begin{lstlisting}
gsettings set com.canonical.indicator.sound visible true
\end{lstlisting}

Reset wifi services in case the connection gets lost.
\begin{lstlisting}
sudo systemctl restart network-manager.service
\end{lstlisting}

Turn off LCD display.
\begin{lstlisting}
xset dpms force off //turns off display.
\end{lstlisting}

Change or view the host name of a computer with the hostname file.
\begin{lstlisting}
sudo nano /etc/hostname # Opens this file using nano for editing.
hostname                # Command to see what the current hostname is.
\end{lstlisting}

Make a file executable and execute a file
\begin{lstlisting}
chmod a+x /location/of/FILE # Makes a file executable
./FILE                      # Executes a file.
\end{lstlisting}

\section{Users and Groups}

Create a new user
\begin{lstlisting}
sudo useradd [options] <USERNAME>             #Creates a user
sudo useradd -e 2016-02-05 <NAME>             #Creates a user that expites on a day.
sudo useradd <USERNAME> -G <GROUPNAME>        #Adds a user to a group upon creation.
useradd --help                                #See full useradd options.
\end{lstlisting}

Change a users password
\begin{lstlisting}
passwd <USERNAME>
\end{lstlisting}

Change the user in terminal
\begin{lstlisting}
su - <USERNAME>
\end{lstlisting}

Add a user to the sudoers group
\begin{lstlisting}
usermod -aG sudo <USERNAME>
\end{lstlisting}













\section{Networking\index{Networking}}

View IP configuration information
\begin{lstlisting}
ifconfig
\end{lstlisting}

Enable/Disable IPv6
\begin{lstlisting}
#Use these two commands to disable IPv6
sudo sysctl -w net.ipv6.conf.all.disable_ipv6=1
sudo sysctl -w net.ipv6.conf.default.disable_ipv6=1

#Use these two commands to re-enable IPv6
sudo sysctl -w net.ipv6.conf.all.disable_ipv6=0
sudo sysctl -w net.ipv6.conf.default.disable_ipv6=0
\end{lstlisting}